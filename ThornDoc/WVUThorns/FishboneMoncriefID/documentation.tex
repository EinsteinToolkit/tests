\documentclass{article}
%\usepackage{../../../../doc/ThornGuide/cactus}
\usepackage{../../../../../doc/latex/cactus}
\newlength{\tableWidth} \newlength{\maxVarWidth} \newlength{\paraWidth} \newlength{\descWidth} \begin{document}

% The title of the document (not necessarily the name of the Thorn)
\title{{\tt FishboneMoncriefID}: Black Hole Accretion Disk Initial Data}

% The author of the documentation - on one line, otherwise it does not work
\author{Author: Zachariah B. Etienne}

% the date your document was last changed, if your document is in CVS, 
% please use:
\date{$ $Date: 2019-09-28 (Sat, 28 Sep 2019) $ $}
\maketitle

% *======================================================================*
%  Cactus Thorn template for ThornGuide documentation
%  Author: Ian Kelley
%  Date: Sun Jun 02, 2002
%  $Header$                                                             
%
%  Thorn documentation in the latex file doc/documentation.tex 
%  will be included in ThornGuides built with the Cactus make system.
%  The scripts employed by the make system automatically include 
%  pages about variables, parameters and scheduling parsed from the 
%  relevent thorn CCL files.
%  
%  This template contains guidelines which help to assure that your     
%  documentation will be correctly added to ThornGuides. More 
%  information is available in the Cactus UsersGuide.
%                                                    
%  Guidelines:
%   - Do not change anything before the line
%       % START CACTUS THORNGUIDE",
%     except for filling in the title, author, date etc. fields.
%        - Each of these fields should only be on ONE line.
%        - Author names should be sparated with a \\ or a comma
%   - You can define your own macros are OK, but they must appear after
%     the START CACTUS THORNGUIDE line, and do not redefine standard 
%     latex commands.
%   - To avoid name clashes with other thorns, 'labels', 'citations', 
%     'references', and 'image' names should conform to the following 
%     convention:          
%       ARRANGEMENT_THORN_LABEL
%     For example, an image wave.eps in the arrangement CactusWave and 
%     thorn WaveToyC should be renamed to CactusWave_WaveToyC_wave.eps
%   - Graphics should only be included using the graphix package. 
%     More specifically, with the "includegraphics" command. Do
%     not specify any graphic file extensions in your .tex file. This 
%     will allow us (later) to create a PDF version of the ThornGuide
%     via pdflatex. |
%   - References should be included with the latex "bibitem" command. 
%   - use \begin{abstract}...\end{abstract} instead of \abstract{...}
%   - For the benefit of our Perl scripts, and for future extensions, 
%     please use simple latex.     
%
% *======================================================================* 
% 
% Example of including a graphic image:
%    \begin{figure}[ht]
%       \begin{center}
%          \includegraphics[width=6cm]{/home/runner/work/tests/tests/arrangements/WVUThorns/FishboneMoncriefID/doc/MyArrangement_MyThorn_MyFigure}
%       \end{center}
%       \caption{Illustration of this and that}
%       \label{MyArrangement_MyThorn_MyLabel}
%    \end{figure}
%
% Example of using a label:
%   \label{MyArrangement_MyThorn_MyLabel}
%
% Example of a citation:
%    \cite{MyArrangement_MyThorn_Author99}
%
% Example of including a reference
%   \bibitem{MyArrangement_MyThorn_Author99}
%   {J. Author, {\em The Title of the Book, Journal, or periodical}, 1 (1999), 
%   1--16. {\tt http://www.nowhere.com/}}
%
% *======================================================================* 

% If you are using CVS use this line to give version information
% $Header$

% Use the Cactus ThornGuide style file
% (Automatically used from Cactus distribution, if you have a 
%  thorn without the Cactus Flesh download this from the Cactus
%  homepage at www.cactuscode.org)

% Do not delete next line
% START CACTUS THORNGUIDE

% Add all definitions used in this documentation here 
%   \def\mydef etc

%\newcommand{\eqref}[1]{(\ref{#1})}

This thorn is documented within two Jupyter notebooks, contained
within the \verb|FishboneMoncriefID/doc/| directory:

\begin{enumerate}
\item \verb|Tutorial-FishboneMoncriefID.ipynb|: Contains basic
  description, equations, and citations for this initial data type.
\item \verb|Tutorial-ETK_thorn-FishboneMoncriefID.ipynb|: Constructs
  this Einstein Toolkit thorn.
\end{enumerate}


% Do not delete next line
% END CACTUS THORNGUIDE



\section{Parameters} 


\parskip = 0pt

\setlength{\tableWidth}{160mm}

\setlength{\paraWidth}{\tableWidth}
\setlength{\descWidth}{\tableWidth}
\settowidth{\maxVarWidth}{FishboneMoncriefID}

\addtolength{\paraWidth}{-\maxVarWidth}
\addtolength{\paraWidth}{-\columnsep}
\addtolength{\paraWidth}{-\columnsep}
\addtolength{\paraWidth}{-\columnsep}

\addtolength{\descWidth}{-\columnsep}
\addtolength{\descWidth}{-\columnsep}
\addtolength{\descWidth}{-\columnsep}
\noindent \begin{tabular*}{\tableWidth}{|c|l@{\extracolsep{\fill}}r|}
\hline
\multicolumn{1}{|p{\maxVarWidth}}{random\_max} & {\bf Scope:} private & REAL \\\hline
\multicolumn{3}{|p{\descWidth}|}{{\bf Description:}   {\em Ceiling value of random perturbation to initial pressure, where perturbed pressure = pressure*(1.0 + (random\_min + (random\_max-random\_min)*RAND[0,1)))}} \\
\hline{\bf Range} & &  {\bf Default:} 0.02 \\\multicolumn{1}{|p{\maxVarWidth}|}{\centering *:*} & \multicolumn{2}{p{\paraWidth}|}{Any value} \\\hline
\end{tabular*}

\vspace{0.5cm}\noindent \begin{tabular*}{\tableWidth}{|c|l@{\extracolsep{\fill}}r|}
\hline
\multicolumn{1}{|p{\maxVarWidth}}{random\_min} & {\bf Scope:} private & REAL \\\hline
\multicolumn{3}{|p{\descWidth}|}{{\bf Description:}   {\em Floor value of random perturbation to initial pressure, where perturbed pressure = pressure*(1.0 + (random\_min + (random\_max-random\_min)*RAND[0,1)))}} \\
\hline{\bf Range} & &  {\bf Default:} -0.02 \\\multicolumn{1}{|p{\maxVarWidth}|}{\centering *:*} & \multicolumn{2}{p{\paraWidth}|}{Any value} \\\hline
\end{tabular*}

\vspace{0.5cm}\noindent \begin{tabular*}{\tableWidth}{|c|l@{\extracolsep{\fill}}r|}
\hline
\multicolumn{1}{|p{\maxVarWidth}}{a} & {\bf Scope:} restricted & REAL \\\hline
\multicolumn{3}{|p{\descWidth}|}{{\bf Description:}   {\em The spin parameter of the black hole}} \\
\hline{\bf Range} & &  {\bf Default:} 0.9375 \\\multicolumn{1}{|p{\maxVarWidth}|}{\centering 0:1.0} & \multicolumn{2}{p{\paraWidth}|}{Positive values, up to 1. Negative disallowed, as certain roots are chosen in the hydro fields setup. Check those before enabling negative spins!} \\\hline
\end{tabular*}

\vspace{0.5cm}\noindent \begin{tabular*}{\tableWidth}{|c|l@{\extracolsep{\fill}}r|}
\hline
\multicolumn{1}{|p{\maxVarWidth}}{a\_b} & {\bf Scope:} restricted & REAL \\\hline
\multicolumn{3}{|p{\descWidth}|}{{\bf Description:}   {\em Scaling factor for the vector potential}} \\
\hline{\bf Range} & &  {\bf Default:} 1.0 \\\multicolumn{1}{|p{\maxVarWidth}|}{\centering *:*} & \multicolumn{2}{p{\paraWidth}|}{} \\\hline
\end{tabular*}

\vspace{0.5cm}\noindent \begin{tabular*}{\tableWidth}{|c|l@{\extracolsep{\fill}}r|}
\hline
\multicolumn{1}{|p{\maxVarWidth}}{gamma} & {\bf Scope:} restricted & REAL \\\hline
\multicolumn{3}{|p{\descWidth}|}{{\bf Description:}   {\em Equation of state: P = kappa * rho\^gamma}} \\
\hline{\bf Range} & &  {\bf Default:} 1.3333333333333333333333333333 \\\multicolumn{1}{|p{\maxVarWidth}|}{\centering 0.0:*} & \multicolumn{2}{p{\paraWidth}|}{Positive values} \\\hline
\end{tabular*}

\vspace{0.5cm}\noindent \begin{tabular*}{\tableWidth}{|c|l@{\extracolsep{\fill}}r|}
\hline
\multicolumn{1}{|p{\maxVarWidth}}{kappa} & {\bf Scope:} restricted & REAL \\\hline
\multicolumn{3}{|p{\descWidth}|}{{\bf Description:}   {\em Equation of state: P = kappa * rho\^gamma}} \\
\hline{\bf Range} & &  {\bf Default:} 1.0e-3 \\\multicolumn{1}{|p{\maxVarWidth}|}{\centering 0.0:*} & \multicolumn{2}{p{\paraWidth}|}{Positive values} \\\hline
\end{tabular*}

\vspace{0.5cm}\noindent \begin{tabular*}{\tableWidth}{|c|l@{\extracolsep{\fill}}r|}
\hline
\multicolumn{1}{|p{\maxVarWidth}}{m} & {\bf Scope:} restricted & REAL \\\hline
\multicolumn{3}{|p{\descWidth}|}{{\bf Description:}   {\em Kerr-Schild BH mass. Probably should always set M=1.}} \\
\hline{\bf Range} & &  {\bf Default:} 1.0 \\\multicolumn{1}{|p{\maxVarWidth}|}{\centering 0.0:*} & \multicolumn{2}{p{\paraWidth}|}{Must be positive} \\\hline
\end{tabular*}

\vspace{0.5cm}\noindent \begin{tabular*}{\tableWidth}{|c|l@{\extracolsep{\fill}}r|}
\hline
\multicolumn{1}{|p{\maxVarWidth}}{r\_at\_max\_density} & {\bf Scope:} restricted & REAL \\\hline
\multicolumn{3}{|p{\descWidth}|}{{\bf Description:}   {\em Radius at maximum disk density. Needs to be {\textgreater} r\_in}} \\
\hline{\bf Range} & &  {\bf Default:} 12.0 \\\multicolumn{1}{|p{\maxVarWidth}|}{\centering 0.0:*} & \multicolumn{2}{p{\paraWidth}|}{Must be positive} \\\hline
\end{tabular*}

\vspace{0.5cm}\noindent \begin{tabular*}{\tableWidth}{|c|l@{\extracolsep{\fill}}r|}
\hline
\multicolumn{1}{|p{\maxVarWidth}}{r\_in} & {\bf Scope:} restricted & REAL \\\hline
\multicolumn{3}{|p{\descWidth}|}{{\bf Description:}   {\em Fixes the inner edge of the disk}} \\
\hline{\bf Range} & &  {\bf Default:} 6.0 \\\multicolumn{1}{|p{\maxVarWidth}|}{\centering 0.0:*} & \multicolumn{2}{p{\paraWidth}|}{Must be positive} \\\hline
\end{tabular*}

\vspace{0.5cm}\noindent \begin{tabular*}{\tableWidth}{|c|l@{\extracolsep{\fill}}r|}
\hline
\multicolumn{1}{|p{\maxVarWidth}}{initial\_hydro} & {\bf Scope:} shared from HYDROBASE & KEYWORD \\\hline
\multicolumn{3}{|l|}{\bf Extends ranges:}\\ 
\hline\multicolumn{1}{|p{\maxVarWidth}|}{\centering FishboneMoncriefID} & \multicolumn{2}{p{\paraWidth}|}{Initial GRHD data from FishboneMoncriefID solution} \\\hline
\end{tabular*}

\vspace{0.5cm}\parskip = 10pt 

\section{Interfaces} 


\parskip = 0pt

\vspace{3mm} \subsection*{General}

\noindent {\bf Implements}: 

fishbonemoncriefid
\vspace{2mm}

\noindent {\bf Inherits}: 

admbase

grid

hydrobase
\vspace{2mm}

\vspace{5mm}\parskip = 10pt 

\section{Schedule} 


\parskip = 0pt


\noindent This section lists all the variables which are assigned storage by thorn WVUThorns/FishboneMoncriefID.  Storage can either last for the duration of the run ({\bf Always} means that if this thorn is activated storage will be assigned, {\bf Conditional} means that if this thorn is activated storage will be assigned for the duration of the run if some condition is met), or can be turned on for the duration of a schedule function.


\subsection*{Storage}

\hspace{5mm}

 \begin{tabular*}{160mm}{ll} 

{\bf Always:}&  ~ \\ 
 ADMBase::metric[metric\_timelevels] ADMBase::curv[metric\_timelevels] ADMBase::lapse[lapse\_timelevels] ADMBase::shift[shift\_timelevels] & ~\\ 
~ & ~\\ 
\end{tabular*} 


\subsection*{Scheduled Functions}
\vspace{5mm}

\noindent {\bf HydroBase\_Initial} 

\hspace{5mm} fishbonemoncrief\_et\_grhd\_initial 

\hspace{5mm}{\it set up general relativistic hydrodynamic (grhd) fields for fishbone-moncrief disk } 


\hspace{5mm}

 \begin{tabular*}{160mm}{cll} 
~ & Language:  & c \\ 
~ & Reads:  & grid::x(everywhere) \\ 
~& ~ &grid::y(everywhere)\\ 
~& ~ &grid::z(everywhere)\\ 
~ & Type:  & function \\ 
~ & Writes:  & admbase::alp(everywhere) \\ 
~& ~ &admbase::betax(everywhere)\\ 
~& ~ &admbase::betay(everywhere)\\ 
~& ~ &admbase::betaz(everywhere)\\ 
~& ~ &admbase::kxx(everywhere)\\ 
~& ~ &admbase::kxy(everywhere)\\ 
~& ~ &admbase::kxz(everywhere)\\ 
~& ~ &admbase::kyy(everywhere)\\ 
~& ~ &admbase::kyz(everywhere)\\ 
~& ~ &admbase::kzz(everywhere)\\ 
~& ~ &admbase::gxx(everywhere)\\ 
~& ~ &admbase::gxy(everywhere)\\ 
~& ~ &admbase::gxz(everywhere)\\ 
~& ~ &admbase::gyy(everywhere)\\ 
~& ~ &admbase::gyz(everywhere)\\ 
~& ~ &admbase::gzz(everywhere)\\ 
~& ~ &hydrobase::vel(everywhere)\\ 
~& ~ &hydrobase::rho(everywhere)\\ 
~& ~ &hydrobase::eps(everywhere)\\ 
~& ~ &hydrobase::press(everywhere)\\ 
\end{tabular*} 


\vspace{5mm}

\noindent {\bf CCTK\_INITIAL} 

\hspace{5mm} fishbonemoncrief\_et\_grhd\_initial\_\_perturb\_pressure 

\hspace{5mm}{\it add random perturbation to initial pressure, after seed magnetic fields have been set up (in case we'd like the seed magnetic fields to depend on the pristine pressures) } 


\hspace{5mm}

 \begin{tabular*}{160mm}{cll} 
~ & After:  & seed\_magnetic\_fields \\ 
~ & Before:  & illinoisgrmhd\_id\_converter \\ 
~ & Language:  & c \\ 
~ & Type:  & function \\ 
\end{tabular*} 



\vspace{5mm}\parskip = 10pt 
\end{document}
