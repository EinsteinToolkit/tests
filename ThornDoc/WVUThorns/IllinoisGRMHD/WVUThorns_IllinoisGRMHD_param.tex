
\section{Parameters} 


\parskip = 0pt

\setlength{\tableWidth}{160mm}

\setlength{\paraWidth}{\tableWidth}
\setlength{\descWidth}{\tableWidth}
\settowidth{\maxVarWidth}{conserv\_to\_prims\_debug}

\addtolength{\paraWidth}{-\maxVarWidth}
\addtolength{\paraWidth}{-\columnsep}
\addtolength{\paraWidth}{-\columnsep}
\addtolength{\paraWidth}{-\columnsep}

\addtolength{\descWidth}{-\columnsep}
\addtolength{\descWidth}{-\columnsep}
\addtolength{\descWidth}{-\columnsep}
\noindent \begin{tabular*}{\tableWidth}{|c|l@{\extracolsep{\fill}}r|}
\hline
\multicolumn{1}{|p{\maxVarWidth}}{damp\_lorenz} & {\bf Scope:} private & REAL \\\hline
\multicolumn{3}{|p{\descWidth}|}{{\bf Description:}   {\em Damping factor for the generalized Lorenz gauge. Has units of 1/length = 1/M. Typically set this parameter to 1.5/(maximum Delta t on AMR grids).}} \\
\hline{\bf Range} & &  {\bf Default:} 0.0 \\\multicolumn{1}{|p{\maxVarWidth}|}{\centering *:*} & \multicolumn{2}{p{\paraWidth}|}{any real} \\\hline
\end{tabular*}

\vspace{0.5cm}\noindent \begin{tabular*}{\tableWidth}{|c|l@{\extracolsep{\fill}}r|}
\hline
\multicolumn{1}{|p{\maxVarWidth}}{conserv\_to\_prims\_debug} & {\bf Scope:} restricted & INT \\\hline
\multicolumn{3}{|p{\descWidth}|}{{\bf Description:}   {\em 0: no, 1: yes}} \\
\hline{\bf Range} & &  {\bf Default:} (none) \\\multicolumn{1}{|p{\maxVarWidth}|}{\centering 0:1} & \multicolumn{2}{p{\paraWidth}|}{zero (no) or one (yes)} \\\hline
\end{tabular*}

\vspace{0.5cm}\noindent \begin{tabular*}{\tableWidth}{|c|l@{\extracolsep{\fill}}r|}
\hline
\multicolumn{1}{|p{\maxVarWidth}}{em\_bc} & {\bf Scope:} restricted & KEYWORD \\\hline
\multicolumn{3}{|p{\descWidth}|}{{\bf Description:}   {\em EM field boundary condition}} \\
\hline{\bf Range} & &  {\bf Default:} copy \\\multicolumn{1}{|p{\maxVarWidth}|}{\centering copy} & \multicolumn{2}{p{\paraWidth}|}{Copy data from nearest boundary point} \\\multicolumn{1}{|p{\maxVarWidth}|}{\centering frozen} & \multicolumn{2}{p{\paraWidth}|}{Frozen boundaries} \\\hline
\end{tabular*}

\vspace{0.5cm}\noindent \begin{tabular*}{\tableWidth}{|c|l@{\extracolsep{\fill}}r|}
\hline
\multicolumn{1}{|p{\maxVarWidth}}{gamma\_speed\_limit} & {\bf Scope:} restricted & REAL \\\hline
\multicolumn{3}{|p{\descWidth}|}{{\bf Description:}   {\em Maximum relativistic gamma factor.}} \\
\hline{\bf Range} & &  {\bf Default:} 10.0 \\\multicolumn{1}{|p{\maxVarWidth}|}{\centering 1:*} & \multicolumn{2}{p{\paraWidth}|}{Positive {\textgreater} 1, though you'll likely have troubles far above 10.} \\\hline
\end{tabular*}

\vspace{0.5cm}\noindent \begin{tabular*}{\tableWidth}{|c|l@{\extracolsep{\fill}}r|}
\hline
\multicolumn{1}{|p{\maxVarWidth}}{gamma\_th} & {\bf Scope:} restricted & REAL \\\hline
\multicolumn{3}{|p{\descWidth}|}{{\bf Description:}   {\em thermal gamma parameter}} \\
\hline{\bf Range} & &  {\bf Default:} -1 \\\multicolumn{1}{|p{\maxVarWidth}|}{\centering 0:*} & \multicolumn{2}{p{\paraWidth}|}{Physical values} \\\multicolumn{1}{|p{\maxVarWidth}|}{\centering -1} & \multicolumn{2}{p{\paraWidth}|}{forbidden value to make sure it is explicitly set in the parfile} \\\hline
\end{tabular*}

\vspace{0.5cm}\noindent \begin{tabular*}{\tableWidth}{|c|l@{\extracolsep{\fill}}r|}
\hline
\multicolumn{1}{|p{\maxVarWidth}}{k\_poly} & {\bf Scope:} restricted & REAL \\\hline
\multicolumn{3}{|p{\descWidth}|}{{\bf Description:}   {\em initial polytropic constant}} \\
\hline{\bf Range} & &  {\bf Default:} 1.0 \\\multicolumn{1}{|p{\maxVarWidth}|}{\centering 0:*} & \multicolumn{2}{p{\paraWidth}|}{Positive} \\\hline
\end{tabular*}

\vspace{0.5cm}\noindent \begin{tabular*}{\tableWidth}{|c|l@{\extracolsep{\fill}}r|}
\hline
\multicolumn{1}{|p{\maxVarWidth}}{matter\_bc} & {\bf Scope:} restricted & KEYWORD \\\hline
\multicolumn{3}{|p{\descWidth}|}{{\bf Description:}   {\em Chosen Matter boundary condition}} \\
\hline{\bf Range} & &  {\bf Default:} outflow \\\multicolumn{1}{|p{\maxVarWidth}|}{\centering outflow} & \multicolumn{2}{p{\paraWidth}|}{Outflow boundary conditions} \\\multicolumn{1}{|p{\maxVarWidth}|}{\centering frozen} & \multicolumn{2}{p{\paraWidth}|}{Frozen boundaries} \\\hline
\end{tabular*}

\vspace{0.5cm}\noindent \begin{tabular*}{\tableWidth}{|c|l@{\extracolsep{\fill}}r|}
\hline
\multicolumn{1}{|p{\maxVarWidth}}{neos} & {\bf Scope:} restricted & INT \\\hline
\multicolumn{3}{|p{\descWidth}|}{{\bf Description:}   {\em number of parameters in EOS table. If you want to increase from the default max value, you MUST also set eos\_params\_arrays1 and eos\_params\_arrays2 in interface.ccl to be consistent!}} \\
\hline{\bf Range} & &  {\bf Default:} 1 \\\multicolumn{1}{|p{\maxVarWidth}|}{\centering 1:10} & \multicolumn{2}{p{\paraWidth}|}{Any integer between 1 and 10} \\\hline
\end{tabular*}

\vspace{0.5cm}\noindent \begin{tabular*}{\tableWidth}{|c|l@{\extracolsep{\fill}}r|}
\hline
\multicolumn{1}{|p{\maxVarWidth}}{psi6threshold} & {\bf Scope:} restricted & REAL \\\hline
\multicolumn{3}{|p{\descWidth}|}{{\bf Description:}   {\em Where Psi\^6 {\textgreater} Psi6threshold, we assume we're inside the horizon in the primitives solver, and certain limits are relaxed or imposed}} \\
\hline{\bf Range} & &  {\bf Default:} 1e100 \\\multicolumn{1}{|p{\maxVarWidth}|}{\centering *:*} & \multicolumn{2}{p{\paraWidth}|}{Can set to anything} \\\hline
\end{tabular*}

\vspace{0.5cm}\noindent \begin{tabular*}{\tableWidth}{|c|l@{\extracolsep{\fill}}r|}
\hline
\multicolumn{1}{|p{\maxVarWidth}}{rho\_b\_atm} & {\bf Scope:} restricted & REAL \\\hline
\multicolumn{3}{|p{\descWidth}|}{{\bf Description:}   {\em Floor value on the baryonic rest mass density rho\_b (atmosphere). Given the variety of systems this code may encounter, there *is no reasonable default*. Your run will die unless you override this default value in your initial data thorn.}} \\
\hline{\bf Range} & &  {\bf Default:} 1e200 \\\multicolumn{1}{|p{\maxVarWidth}|}{\centering *:*} & \multicolumn{2}{p{\paraWidth}|}{Allow for negative values.  This enables us to debug the code and verify if rho\_b\_atm is properly set.} \\\hline
\end{tabular*}

\vspace{0.5cm}\noindent \begin{tabular*}{\tableWidth}{|c|l@{\extracolsep{\fill}}r|}
\hline
\multicolumn{1}{|p{\maxVarWidth}}{rho\_b\_max} & {\bf Scope:} restricted & REAL \\\hline
\multicolumn{3}{|p{\descWidth}|}{{\bf Description:}   {\em Ceiling value on the baryonic rest mass density rho\_b. The enormous value effectively disables this ceiling by default. It can be quite useful after a black hole has accreted a lot of mass, leading to enormous densities inside the BH. To enable this trick, set rho\_b\_max in your initial data thorn! You are welcome to change this parameter mid-run (after restarting from a checkpoint).}} \\
\hline{\bf Range} & &  {\bf Default:} 1e300 \\\multicolumn{1}{|p{\maxVarWidth}|}{\centering 0:*} & \multicolumn{2}{p{\paraWidth}|}{Note that you will have problems unless rho\_b\_atm{\textless}rho\_b\_max} \\\hline
\end{tabular*}

\vspace{0.5cm}\noindent \begin{tabular*}{\tableWidth}{|c|l@{\extracolsep{\fill}}r|}
\hline
\multicolumn{1}{|p{\maxVarWidth}}{sym\_bz} & {\bf Scope:} restricted & REAL \\\hline
\multicolumn{3}{|p{\descWidth}|}{{\bf Description:}   {\em In-progress equatorial symmetry support: Symmetry parameter across z axis for magnetic fields = +/- 1}} \\
\hline{\bf Range} & &  {\bf Default:} 1.0 \\\multicolumn{1}{|p{\maxVarWidth}|}{\centering -1.0:1.0} & \multicolumn{2}{p{\paraWidth}|}{Set to +1 or -1.} \\\hline
\end{tabular*}

\vspace{0.5cm}\noindent \begin{tabular*}{\tableWidth}{|c|l@{\extracolsep{\fill}}r|}
\hline
\multicolumn{1}{|p{\maxVarWidth}}{symmetry} & {\bf Scope:} restricted & KEYWORD \\\hline
\multicolumn{3}{|p{\descWidth}|}{{\bf Description:}   {\em Currently only no symmetry supported, though work has begun in adding equatorial-symmetry support. FIXME: Extend ET symmetry interface to support symmetries on staggered gridfunctions}} \\
\hline{\bf Range} & &  {\bf Default:} none \\\multicolumn{1}{|p{\maxVarWidth}|}{\centering none} & \multicolumn{2}{p{\paraWidth}|}{no symmetry, full 3d domain} \\\hline
\end{tabular*}

\vspace{0.5cm}\noindent \begin{tabular*}{\tableWidth}{|c|l@{\extracolsep{\fill}}r|}
\hline
\multicolumn{1}{|p{\maxVarWidth}}{tau\_atm} & {\bf Scope:} restricted & REAL \\\hline
\multicolumn{3}{|p{\descWidth}|}{{\bf Description:}   {\em Floor value on the energy variable tau (cf. tau\_stildefix\_enable). Given the variety of systems this code may encounter, there *is no reasonable default*. Effectively the current (enormous) value should disable the tau\_atm floor. Please set this in your initial data thorn, and reset at will during evolutions.}} \\
\hline{\bf Range} & &  {\bf Default:} 1e100 \\\multicolumn{1}{|p{\maxVarWidth}|}{\centering 0:*} & \multicolumn{2}{p{\paraWidth}|}{Positive} \\\hline
\end{tabular*}

\vspace{0.5cm}\noindent \begin{tabular*}{\tableWidth}{|c|l@{\extracolsep{\fill}}r|}
\hline
\multicolumn{1}{|p{\maxVarWidth}}{update\_tmunu} & {\bf Scope:} restricted & BOOLEAN \\\hline
\multicolumn{3}{|p{\descWidth}|}{{\bf Description:}   {\em Update Tmunu, for RHS of Einstein's equations?}} \\
\hline & & {\bf Default:} yes \\\hline
\end{tabular*}

\vspace{0.5cm}\noindent \begin{tabular*}{\tableWidth}{|c|l@{\extracolsep{\fill}}r|}
\hline
\multicolumn{1}{|p{\maxVarWidth}}{verbose} & {\bf Scope:} restricted & KEYWORD \\\hline
\multicolumn{3}{|p{\descWidth}|}{{\bf Description:}   {\em Determines how much evolution information is output}} \\
\hline{\bf Range} & &  {\bf Default:} essential+iteration output \\\multicolumn{1}{|p{\maxVarWidth}|}{\centering no} & \multicolumn{2}{p{\paraWidth}|}{Complete silence} \\\multicolumn{1}{|p{\maxVarWidth}|}{\centering essential} & \multicolumn{2}{p{\paraWidth}|}{"Essential health monitoring of the GRMHD evolution: Information about conservative-to-prim 
itive fixes, etc."} \\\multicolumn{1}{|p{\maxVarWidth}|}{see [1] below} & \multicolumn{2}{p{\paraWidth}|}{Outputs health monitoring information, plus a record of which RK iteration. Very useful for backtracing a crash.} \\\hline
\end{tabular*}

\vspace{0.5cm}\noindent {\bf [1]} \noindent \begin{verbatim}essential+iteration output\end{verbatim}\noindent \begin{tabular*}{\tableWidth}{|c|l@{\extracolsep{\fill}}r|}
\hline
\multicolumn{1}{|p{\maxVarWidth}}{lapse\_timelevels} & {\bf Scope:} shared from ADMBASE & INT \\\hline
\end{tabular*}

\vspace{0.5cm}\noindent \begin{tabular*}{\tableWidth}{|c|l@{\extracolsep{\fill}}r|}
\hline
\multicolumn{1}{|p{\maxVarWidth}}{metric\_timelevels} & {\bf Scope:} shared from ADMBASE & INT \\\hline
\end{tabular*}

\vspace{0.5cm}\noindent \begin{tabular*}{\tableWidth}{|c|l@{\extracolsep{\fill}}r|}
\hline
\multicolumn{1}{|p{\maxVarWidth}}{shift\_timelevels} & {\bf Scope:} shared from ADMBASE & INT \\\hline
\end{tabular*}

\vspace{0.5cm}\parskip = 10pt 
