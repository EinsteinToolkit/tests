\documentclass{article}

% Use the Cactus ThornGuide style file
% (Automatically used from Cactus distribution, if you have a 
%  thorn without the Cactus Flesh download this from the Cactus
%  homepage at www.cactuscode.org)
\usepackage{../../../../../doc/latex/cactus}

\newlength{\tableWidth} \newlength{\maxVarWidth} \newlength{\paraWidth} \newlength{\descWidth} \begin{document}

\title{GRHayLID}
\author{Samuel Cupp \\ Leonardo Rosa Werneck}
\date{$ $Date$ $}

\maketitle

% Do not delete next line
% START CACTUS THORNGUIDE

% Add all definitions used in this documentation here
\newcommand{\glib}{\texttt{GRHayLib}}
\newcommand{\gid}{\texttt{GRHayLID}}
\newcommand{\hbase}{\texttt{HydroBase}}

\begin{abstract}
Provides simple magnetohydrodynamic (MHD) initial data for testing,
as well as functionality for imposing beta equilibrium on initial
hydrodynamic data and computing the initial entropy. The test data
includes the Balsara tests, a cylindrical explosion, etc. While
some initial data requires EOS information provided by \glib,
the data can then be used by any evolution thorn so long as their EOS
framework is set up to match \glib.
\end{abstract}

\section{Introduction}

This thorn provides a variety of initial data setups for
testing GRMHD evolution codes. For 1D, this includes all five
Balsara tests, equilibrium initial data, a sound wave, and
a shock tube. For 3D, we provide isotropic gas and
constant-density sphere initial data. We also provide the option
to impose beta-equilibrium on provided initial data.

The initial data is separated into magnetic and hydrodynamic
parts, so the magnetic initial data can be turned off if desired.
All settings use the \hbase{} variables for compatibility with
other thorns, but only Avec is set. The Aphi or B variables are
not set by this thorn.

\section{Parameters}

The parameters are mostly self-explanatory or just extensions of
\hbase{} keywords. The param.ccl is divided into sections based
on which tests the parameters are for, but we explain them here.
The type of test is controlled by \textbf{initial\_hydro} from
\hbase. All the options for these parameters are explicitly
listed by the autogenerated parameter documentation below.

Note that neither beta equilibrium nor entropy calculation are
tied to \textbf{initial\_hydro}, so they can be used even if
the actual initial data is set by another thorn. Beta equilibrium
is controlled by the \textbf{impose\_beta\_equilibrium} parameter,
and entropy calculation controlled by the \textbf{compute\_entropy}
parameter.

Some initial data is only valid for tabulated or hybrid EOS, and
these check the EOS parameter from \glib. If it is set incorrectly,
the run will end with an error message stating the incompatibility.
They will also check if \hbase{} parameters are set in a compatible
fashion (such as \texttt{HydroBase::initial\_temperature}) and
will also error out if these are set incorrectly.

\subsection{HydroTest1D}

This option is used for all the 1D tests we provide. The
\textbf{initial\_data\_1D} parameter chooses the
specific test and defaults to the Balsara1 test. For shocks and
waves, the \textbf{shock\_direction} parameter changes the
direction of the shock, and the \textbf{discontinuity\_position}
parameter changes the location of the shock front. For the sound
wave test, the amplitude can be set with \textbf{wave\_amplitude}.

\subsection{IsotropicGas}

This option sets the hydrodynamic variables to an isotropic gas using
the tabulated equation of state and does not work with other EOS
settings. It does not set magnetic quantities. The thorn checks to
ensure parameters are set correctly in \glib and
\hbase so that the initial data works, and the run will end
with an error message if the core parameters are misconfigured.

Specifically, the thorn checks the following parameters:
\begin{tabular}{l|l}
Thorn \& Parameter & Correct Setting \\\hline
\texttt{GRHayLib::EOS\_type} & "tabulated" \\
\texttt{HydroBase::initial\_Y\_e} & "GRHayLID" \\
\texttt{HydroBase::initial\_temperature} & "GRHayLID"
\end{tabular}

\subsection{ConstantDensitySphere}

This option sets the hydrodynamic variables to a sphere of constant
density using the tabulated equation of state and does not work
with other EOS settings. It does not set magnetic quantities. The thorn
checks to ensure parameters are set correctly in \glib and
\hbase so that the initial data works, and the run will end
with an error message if the core parameters are misconfigured.

Specifically, the thorn checks the following parameters:
\begin{tabular}{l|l}
Thorn \& Parameter & Correct Setting \\\hline
\texttt{GRHayLib::EOS\_type} & "tabulated" \\
\texttt{HydroBase::initial\_Y\_e} & "GRHayLID" \\
\texttt{HydroBase::initial\_temperature} & "GRHayLID"
\end{tabular}

% Do not delete next line
% END CACTUS THORNGUIDE



\section{Parameters} 


\parskip = 0pt

\setlength{\tableWidth}{160mm}

\setlength{\paraWidth}{\tableWidth}
\setlength{\descWidth}{\tableWidth}
\settowidth{\maxVarWidth}{constantdensitysphere\_sphere\_radius}

\addtolength{\paraWidth}{-\maxVarWidth}
\addtolength{\paraWidth}{-\columnsep}
\addtolength{\paraWidth}{-\columnsep}
\addtolength{\paraWidth}{-\columnsep}

\addtolength{\descWidth}{-\columnsep}
\addtolength{\descWidth}{-\columnsep}
\addtolength{\descWidth}{-\columnsep}
\noindent \begin{tabular*}{\tableWidth}{|c|l@{\extracolsep{\fill}}r|}
\hline
\multicolumn{1}{|p{\maxVarWidth}}{beq\_temperature} & {\bf Scope:} restricted & REAL \\\hline
\multicolumn{3}{|p{\descWidth}|}{{\bf Description:}   {\em Temperature used to impose beta-equilibrium}} \\
\hline{\bf Range} & &  {\bf Default:} -1 \\\multicolumn{1}{|p{\maxVarWidth}|}{\centering 0:*} & \multicolumn{2}{p{\paraWidth}|}{Non-negative values} \\\multicolumn{1}{|p{\maxVarWidth}|}{\centering -1} & \multicolumn{2}{p{\paraWidth}|}{Forbidden value} \\\hline
\end{tabular*}

\vspace{0.5cm}\noindent \begin{tabular*}{\tableWidth}{|c|l@{\extracolsep{\fill}}r|}
\hline
\multicolumn{1}{|p{\maxVarWidth}}{constantdensitysphere\_rho\_exterior} & {\bf Scope:} restricted & REAL \\\hline
\multicolumn{3}{|p{\descWidth}|}{{\bf Description:}   {\em Exterior density in geometrized units}} \\
\hline{\bf Range} & &  {\bf Default:} -1 \\\multicolumn{1}{|p{\maxVarWidth}|}{\centering 0:*} & \multicolumn{2}{p{\paraWidth}|}{Non-negative values} \\\multicolumn{1}{|p{\maxVarWidth}|}{\centering -1} & \multicolumn{2}{p{\paraWidth}|}{Forbidden value} \\\hline
\end{tabular*}

\vspace{0.5cm}\noindent \begin{tabular*}{\tableWidth}{|c|l@{\extracolsep{\fill}}r|}
\hline
\multicolumn{1}{|p{\maxVarWidth}}{constantdensitysphere\_rho\_interior} & {\bf Scope:} restricted & REAL \\\hline
\multicolumn{3}{|p{\descWidth}|}{{\bf Description:}   {\em Interior density in geometrized units}} \\
\hline{\bf Range} & &  {\bf Default:} -1 \\\multicolumn{1}{|p{\maxVarWidth}|}{\centering 0:*} & \multicolumn{2}{p{\paraWidth}|}{Non-negative values} \\\multicolumn{1}{|p{\maxVarWidth}|}{\centering -1} & \multicolumn{2}{p{\paraWidth}|}{Forbidden value} \\\hline
\end{tabular*}

\vspace{0.5cm}\noindent \begin{tabular*}{\tableWidth}{|c|l@{\extracolsep{\fill}}r|}
\hline
\multicolumn{1}{|p{\maxVarWidth}}{constantdensitysphere\_sphere\_radius} & {\bf Scope:} restricted & REAL \\\hline
\multicolumn{3}{|p{\descWidth}|}{{\bf Description:}   {\em Sphere radius in geometrized units}} \\
\hline{\bf Range} & &  {\bf Default:} -1 \\\multicolumn{1}{|p{\maxVarWidth}|}{\centering 0:*} & \multicolumn{2}{p{\paraWidth}|}{Non-negative values} \\\multicolumn{1}{|p{\maxVarWidth}|}{\centering -1} & \multicolumn{2}{p{\paraWidth}|}{Forbidden value} \\\hline
\end{tabular*}

\vspace{0.5cm}\noindent \begin{tabular*}{\tableWidth}{|c|l@{\extracolsep{\fill}}r|}
\hline
\multicolumn{1}{|p{\maxVarWidth}}{constantdensitysphere\_t\_exterior} & {\bf Scope:} restricted & REAL \\\hline
\multicolumn{3}{|p{\descWidth}|}{{\bf Description:}   {\em Exterior temperature in MeV}} \\
\hline{\bf Range} & &  {\bf Default:} -1 \\\multicolumn{1}{|p{\maxVarWidth}|}{\centering 0:*} & \multicolumn{2}{p{\paraWidth}|}{Non-negative values} \\\multicolumn{1}{|p{\maxVarWidth}|}{\centering -1} & \multicolumn{2}{p{\paraWidth}|}{Forbidden value} \\\hline
\end{tabular*}

\vspace{0.5cm}\noindent \begin{tabular*}{\tableWidth}{|c|l@{\extracolsep{\fill}}r|}
\hline
\multicolumn{1}{|p{\maxVarWidth}}{constantdensitysphere\_t\_interior} & {\bf Scope:} restricted & REAL \\\hline
\multicolumn{3}{|p{\descWidth}|}{{\bf Description:}   {\em Interior temperature in MeV}} \\
\hline{\bf Range} & &  {\bf Default:} -1 \\\multicolumn{1}{|p{\maxVarWidth}|}{\centering 0:*} & \multicolumn{2}{p{\paraWidth}|}{Non-negative values} \\\multicolumn{1}{|p{\maxVarWidth}|}{\centering -1} & \multicolumn{2}{p{\paraWidth}|}{Forbidden value} \\\hline
\end{tabular*}

\vspace{0.5cm}\noindent \begin{tabular*}{\tableWidth}{|c|l@{\extracolsep{\fill}}r|}
\hline
\multicolumn{1}{|p{\maxVarWidth}}{constantdensitysphere\_vx\_interior} & {\bf Scope:} restricted & REAL \\\hline
\multicolumn{3}{|p{\descWidth}|}{{\bf Description:}   {\em Interior x-velocity}} \\
\hline{\bf Range} & &  {\bf Default:} (none) \\\multicolumn{1}{|p{\maxVarWidth}|}{\centering *:*} & \multicolumn{2}{p{\paraWidth}|}{Anything goes} \\\hline
\end{tabular*}

\vspace{0.5cm}\noindent \begin{tabular*}{\tableWidth}{|c|l@{\extracolsep{\fill}}r|}
\hline
\multicolumn{1}{|p{\maxVarWidth}}{constantdensitysphere\_vy\_interior} & {\bf Scope:} restricted & REAL \\\hline
\multicolumn{3}{|p{\descWidth}|}{{\bf Description:}   {\em Interior y-velocity}} \\
\hline{\bf Range} & &  {\bf Default:} (none) \\\multicolumn{1}{|p{\maxVarWidth}|}{\centering *:*} & \multicolumn{2}{p{\paraWidth}|}{Anything goes} \\\hline
\end{tabular*}

\vspace{0.5cm}\noindent \begin{tabular*}{\tableWidth}{|c|l@{\extracolsep{\fill}}r|}
\hline
\multicolumn{1}{|p{\maxVarWidth}}{constantdensitysphere\_vz\_interior} & {\bf Scope:} restricted & REAL \\\hline
\multicolumn{3}{|p{\descWidth}|}{{\bf Description:}   {\em Interior z-velocity}} \\
\hline{\bf Range} & &  {\bf Default:} (none) \\\multicolumn{1}{|p{\maxVarWidth}|}{\centering *:*} & \multicolumn{2}{p{\paraWidth}|}{Anything goes} \\\hline
\end{tabular*}

\vspace{0.5cm}\noindent \begin{tabular*}{\tableWidth}{|c|l@{\extracolsep{\fill}}r|}
\hline
\multicolumn{1}{|p{\maxVarWidth}}{constantdensitysphere\_y\_e\_exterior} & {\bf Scope:} restricted & REAL \\\hline
\multicolumn{3}{|p{\descWidth}|}{{\bf Description:}   {\em Exterior electron fraction (dimensionless)}} \\
\hline{\bf Range} & &  {\bf Default:} -1 \\\multicolumn{1}{|p{\maxVarWidth}|}{\centering 0:*} & \multicolumn{2}{p{\paraWidth}|}{Non-negative values} \\\multicolumn{1}{|p{\maxVarWidth}|}{\centering -1} & \multicolumn{2}{p{\paraWidth}|}{Forbidden value} \\\hline
\end{tabular*}

\vspace{0.5cm}\noindent \begin{tabular*}{\tableWidth}{|c|l@{\extracolsep{\fill}}r|}
\hline
\multicolumn{1}{|p{\maxVarWidth}}{constantdensitysphere\_y\_e\_interior} & {\bf Scope:} restricted & REAL \\\hline
\multicolumn{3}{|p{\descWidth}|}{{\bf Description:}   {\em Interior electron fraction (dimensionless)}} \\
\hline{\bf Range} & &  {\bf Default:} -1 \\\multicolumn{1}{|p{\maxVarWidth}|}{\centering 0:*} & \multicolumn{2}{p{\paraWidth}|}{Non-negative values} \\\multicolumn{1}{|p{\maxVarWidth}|}{\centering -1} & \multicolumn{2}{p{\paraWidth}|}{Forbidden value} \\\hline
\end{tabular*}

\vspace{0.5cm}\noindent \begin{tabular*}{\tableWidth}{|c|l@{\extracolsep{\fill}}r|}
\hline
\multicolumn{1}{|p{\maxVarWidth}}{discontinuity\_position} & {\bf Scope:} restricted & REAL \\\hline
\multicolumn{3}{|p{\descWidth}|}{{\bf Description:}   {\em Set location of the shock discontinuity}} \\
\hline{\bf Range} & &  {\bf Default:} 0.0 \\\multicolumn{1}{|p{\maxVarWidth}|}{\centering *:*} & \multicolumn{2}{p{\paraWidth}|}{} \\\hline
\end{tabular*}

\vspace{0.5cm}\noindent \begin{tabular*}{\tableWidth}{|c|l@{\extracolsep{\fill}}r|}
\hline
\multicolumn{1}{|p{\maxVarWidth}}{impose\_beta\_equilibrium} & {\bf Scope:} restricted & BOOLEAN \\\hline
\multicolumn{3}{|p{\descWidth}|}{{\bf Description:}   {\em Impose beta-equilibrium on initial data}} \\
\hline & & {\bf Default:} no \\\hline
\end{tabular*}

\vspace{0.5cm}\noindent \begin{tabular*}{\tableWidth}{|c|l@{\extracolsep{\fill}}r|}
\hline
\multicolumn{1}{|p{\maxVarWidth}}{initial\_data\_1d} & {\bf Scope:} restricted & KEYWORD \\\hline
\multicolumn{3}{|p{\descWidth}|}{{\bf Description:}   {\em Set initial data type for 1D tests}} \\
\hline{\bf Range} & &  {\bf Default:} Balsara1 \\\multicolumn{1}{|p{\maxVarWidth}|}{\centering equilibrium} & \multicolumn{2}{p{\paraWidth}|}{initial data for the equilibrium test} \\\multicolumn{1}{|p{\maxVarWidth}|}{\centering sound wave} & \multicolumn{2}{p{\paraWidth}|}{initial data for the sound wave test} \\\multicolumn{1}{|p{\maxVarWidth}|}{\centering shock tube} & \multicolumn{2}{p{\paraWidth}|}{initial data for the shock tube test} \\\multicolumn{1}{|p{\maxVarWidth}|}{\centering Balsara1} & \multicolumn{2}{p{\paraWidth}|}{initial data for the Balsara1 test} \\\multicolumn{1}{|p{\maxVarWidth}|}{\centering Balsara2} & \multicolumn{2}{p{\paraWidth}|}{initial data for the Balsara2 test} \\\multicolumn{1}{|p{\maxVarWidth}|}{\centering Balsara3} & \multicolumn{2}{p{\paraWidth}|}{initial data for the Balsara3 test} \\\multicolumn{1}{|p{\maxVarWidth}|}{\centering Balsara4} & \multicolumn{2}{p{\paraWidth}|}{initial data for the Balsara4 test} \\\multicolumn{1}{|p{\maxVarWidth}|}{\centering Balsara5} & \multicolumn{2}{p{\paraWidth}|}{initial data for the Balsara5 test} \\\hline
\end{tabular*}

\vspace{0.5cm}\noindent \begin{tabular*}{\tableWidth}{|c|l@{\extracolsep{\fill}}r|}
\hline
\multicolumn{1}{|p{\maxVarWidth}}{initialize\_magnetic\_quantities} & {\bf Scope:} restricted & BOOLEAN \\\hline
\multicolumn{3}{|p{\descWidth}|}{{\bf Description:}   {\em Set whether Avec should be initialized}} \\
\hline & & {\bf Default:} yes \\\hline
\end{tabular*}

\vspace{0.5cm}\noindent \begin{tabular*}{\tableWidth}{|c|l@{\extracolsep{\fill}}r|}
\hline
\multicolumn{1}{|p{\maxVarWidth}}{isotropicgas\_rho} & {\bf Scope:} restricted & REAL \\\hline
\multicolumn{3}{|p{\descWidth}|}{{\bf Description:}   {\em Gas density in geometrized units}} \\
\hline{\bf Range} & &  {\bf Default:} -1 \\\multicolumn{1}{|p{\maxVarWidth}|}{\centering 0:*} & \multicolumn{2}{p{\paraWidth}|}{Non-negative values} \\\multicolumn{1}{|p{\maxVarWidth}|}{\centering -1} & \multicolumn{2}{p{\paraWidth}|}{Forbidden value} \\\hline
\end{tabular*}

\vspace{0.5cm}\noindent \begin{tabular*}{\tableWidth}{|c|l@{\extracolsep{\fill}}r|}
\hline
\multicolumn{1}{|p{\maxVarWidth}}{isotropicgas\_temperature} & {\bf Scope:} restricted & REAL \\\hline
\multicolumn{3}{|p{\descWidth}|}{{\bf Description:}   {\em Gas temperature in MeV}} \\
\hline{\bf Range} & &  {\bf Default:} -1 \\\multicolumn{1}{|p{\maxVarWidth}|}{\centering 0:*} & \multicolumn{2}{p{\paraWidth}|}{Non-negative values} \\\multicolumn{1}{|p{\maxVarWidth}|}{\centering -1} & \multicolumn{2}{p{\paraWidth}|}{Forbidden value} \\\hline
\end{tabular*}

\vspace{0.5cm}\noindent \begin{tabular*}{\tableWidth}{|c|l@{\extracolsep{\fill}}r|}
\hline
\multicolumn{1}{|p{\maxVarWidth}}{isotropicgas\_y\_e} & {\bf Scope:} restricted & REAL \\\hline
\multicolumn{3}{|p{\descWidth}|}{{\bf Description:}   {\em Gas electron fraction (dimensionless)}} \\
\hline{\bf Range} & &  {\bf Default:} -1 \\\multicolumn{1}{|p{\maxVarWidth}|}{\centering 0:*} & \multicolumn{2}{p{\paraWidth}|}{Non-negative values} \\\multicolumn{1}{|p{\maxVarWidth}|}{\centering -1} & \multicolumn{2}{p{\paraWidth}|}{Forbidden value} \\\hline
\end{tabular*}

\vspace{0.5cm}\noindent \begin{tabular*}{\tableWidth}{|c|l@{\extracolsep{\fill}}r|}
\hline
\multicolumn{1}{|p{\maxVarWidth}}{shock\_direction} & {\bf Scope:} restricted & KEYWORD \\\hline
\multicolumn{3}{|p{\descWidth}|}{{\bf Description:}   {\em Set direction for shock in test data}} \\
\hline{\bf Range} & &  {\bf Default:} x \\\multicolumn{1}{|p{\maxVarWidth}|}{\centering x} & \multicolumn{2}{p{\paraWidth}|}{Test data will have discontinuities along x direction} \\\multicolumn{1}{|p{\maxVarWidth}|}{\centering y} & \multicolumn{2}{p{\paraWidth}|}{Test data will have discontinuities along y direction} \\\multicolumn{1}{|p{\maxVarWidth}|}{\centering z} & \multicolumn{2}{p{\paraWidth}|}{Test data will have discontinuities along z direction} \\\hline
\end{tabular*}

\vspace{0.5cm}\noindent \begin{tabular*}{\tableWidth}{|c|l@{\extracolsep{\fill}}r|}
\hline
\multicolumn{1}{|p{\maxVarWidth}}{stagger\_a\_fields} & {\bf Scope:} restricted & BOOLEAN \\\hline
\multicolumn{3}{|p{\descWidth}|}{{\bf Description:}   {\em Sets whether Avec is staggered to +1/2 (following IllinoisGRMHD convention)}} \\
\hline & & {\bf Default:} yes \\\hline
\end{tabular*}

\vspace{0.5cm}\noindent \begin{tabular*}{\tableWidth}{|c|l@{\extracolsep{\fill}}r|}
\hline
\multicolumn{1}{|p{\maxVarWidth}}{wave\_amplitude} & {\bf Scope:} restricted & REAL \\\hline
\multicolumn{3}{|p{\descWidth}|}{{\bf Description:}   {\em Wave amplitude for sound wave initial data}} \\
\hline{\bf Range} & &  {\bf Default:} 1.0e-3 \\\multicolumn{1}{|p{\maxVarWidth}|}{\centering 0.0:*} & \multicolumn{2}{p{\paraWidth}|}{} \\\hline
\end{tabular*}

\vspace{0.5cm}\noindent \begin{tabular*}{\tableWidth}{|c|l@{\extracolsep{\fill}}r|}
\hline
\multicolumn{1}{|p{\maxVarWidth}}{initial\_avec} & {\bf Scope:} shared from HYDROBASE & KEYWORD \\\hline
\multicolumn{3}{|l|}{\bf Extends ranges:}\\ 
\hline\multicolumn{1}{|p{\maxVarWidth}|}{\centering GRHayLID} & \multicolumn{2}{p{\paraWidth}|}{Initial vector potential set by GRHayLID} \\\hline
\end{tabular*}

\vspace{0.5cm}\noindent \begin{tabular*}{\tableWidth}{|c|l@{\extracolsep{\fill}}r|}
\hline
\multicolumn{1}{|p{\maxVarWidth}}{initial\_bvec} & {\bf Scope:} shared from HYDROBASE & KEYWORD \\\hline
\multicolumn{3}{|l|}{\bf Extends ranges:}\\ 
\hline\multicolumn{1}{|p{\maxVarWidth}|}{\centering GRHayLID} & \multicolumn{2}{p{\paraWidth}|}{Initial magnetic fields set by GRHayLID} \\\hline
\end{tabular*}

\vspace{0.5cm}\noindent \begin{tabular*}{\tableWidth}{|c|l@{\extracolsep{\fill}}r|}
\hline
\multicolumn{1}{|p{\maxVarWidth}}{initial\_entropy} & {\bf Scope:} shared from HYDROBASE & KEYWORD \\\hline
\multicolumn{3}{|l|}{\bf Extends ranges:}\\ 
\hline\multicolumn{1}{|p{\maxVarWidth}|}{\centering GRHayLID} & \multicolumn{2}{p{\paraWidth}|}{Initial entropy set by GRHayLID} \\\hline
\end{tabular*}

\vspace{0.5cm}\noindent \begin{tabular*}{\tableWidth}{|c|l@{\extracolsep{\fill}}r|}
\hline
\multicolumn{1}{|p{\maxVarWidth}}{initial\_hydro} & {\bf Scope:} shared from HYDROBASE & KEYWORD \\\hline
\multicolumn{3}{|l|}{\bf Extends ranges:}\\ 
\hline\multicolumn{1}{|p{\maxVarWidth}|}{\centering HydroTest1D} & \multicolumn{2}{p{\paraWidth}|}{"Initial data set by 1D test; see test\_1D\_initial\_data 
 parameter for options"} \\\multicolumn{1}{|p{\maxVarWidth}|}{\centering IsotropicGas} & \multicolumn{2}{p{\paraWidth}|}{Initial data set to isotropic gas} \\\multicolumn{1}{|p{\maxVarWidth}|}{see [1] below} & \multicolumn{2}{p{\paraWidth}|}{Initial data set to constant density sphere} \\\hline
\end{tabular*}

\vspace{0.5cm}\noindent {\bf [1]} \noindent \begin{verbatim}ConstantDensitySphere\end{verbatim}\noindent \begin{tabular*}{\tableWidth}{|c|l@{\extracolsep{\fill}}r|}
\hline
\multicolumn{1}{|p{\maxVarWidth}}{initial\_temperature} & {\bf Scope:} shared from HYDROBASE & KEYWORD \\\hline
\multicolumn{3}{|l|}{\bf Extends ranges:}\\ 
\hline\multicolumn{1}{|p{\maxVarWidth}|}{\centering GRHayLID} & \multicolumn{2}{p{\paraWidth}|}{Initial temperature set by GRHayLID} \\\hline
\end{tabular*}

\vspace{0.5cm}\noindent \begin{tabular*}{\tableWidth}{|c|l@{\extracolsep{\fill}}r|}
\hline
\multicolumn{1}{|p{\maxVarWidth}}{initial\_y\_e} & {\bf Scope:} shared from HYDROBASE & KEYWORD \\\hline
\multicolumn{3}{|l|}{\bf Extends ranges:}\\ 
\hline\multicolumn{1}{|p{\maxVarWidth}|}{\centering GRHayLID} & \multicolumn{2}{p{\paraWidth}|}{Initial Y\_e set by GRHayLID} \\\hline
\end{tabular*}

\vspace{0.5cm}\parskip = 10pt 

\section{Interfaces} 


\parskip = 0pt

\vspace{3mm} \subsection*{General}

\noindent {\bf Implements}: 

grhaylid
\vspace{2mm}

\noindent {\bf Inherits}: 

grhaylib

grid

hydrobase
\vspace{2mm}

\vspace{5mm}

\noindent {\bf Uses header}: 

GRHayLib.h
\vspace{2mm}\parskip = 10pt 

\section{Schedule} 


\parskip = 0pt


\noindent This section lists all the variables which are assigned storage by thorn GRHayLET/GRHayLID.  Storage can either last for the duration of the run ({\bf Always} means that if this thorn is activated storage will be assigned, {\bf Conditional} means that if this thorn is activated storage will be assigned for the duration of the run if some condition is met), or can be turned on for the duration of a schedule function.


\subsection*{Storage}NONE
\subsection*{Scheduled Functions}
\vspace{5mm}

\noindent {\bf HydroBase\_Initial}   (conditional) 

\hspace{5mm} grhaylid\_1d\_tests\_hydro\_data 

\hspace{5mm}{\it set initial hydrodynamic data for 1d test } 


\hspace{5mm}

 \begin{tabular*}{160mm}{cll} 
~ & Language:  & c \\ 
~ & Reads:  & grid::coordinates \\ 
~ & Type:  & function \\ 
~ & Writes:  & hydrobase::rho(everywhere) \\ 
~& ~ &hydrobase::press(everywhere)\\ 
~& ~ &hydrobase::eps(everywhere)\\ 
~& ~ &hydrobase::vel(everywhere)\\ 
\end{tabular*} 


\vspace{5mm}

\noindent {\bf HydroBase\_Initial}   (conditional) 

\hspace{5mm} grhaylid\_1d\_tests\_magnetic\_data 

\hspace{5mm}{\it set initial magnetic data for 1d test } 


\hspace{5mm}

 \begin{tabular*}{160mm}{cll} 
~ & After:  & grhaylid\_1d\_tests\_hydro\_data \\ 
~ & Language:  & c \\ 
~ & Reads:  & grid::coordinates \\ 
~ & Type:  & function \\ 
~ & Writes:  & hydrobase::avec(everywhere) \\ 
~& ~ &hydrobase::bvec(everywhere)\\ 
\end{tabular*} 


\vspace{5mm}

\noindent {\bf HydroBase\_Initial}   (conditional) 

\hspace{5mm} grhaylid\_isotropicgas 

\hspace{5mm}{\it set initial hydrodynamic data for 1d test } 


\hspace{5mm}

 \begin{tabular*}{160mm}{cll} 
~ & Language:  & c \\ 
~ & Reads:  & grid::coordinates \\ 
~ & Type:  & function \\ 
~ & Writes:  & hydrobase::rho(everywhere) \\ 
~& ~ &hydrobase::press(everywhere)\\ 
~& ~ &hydrobase::eps(everywhere)\\ 
~& ~ &hydrobase::vel(everywhere)\\ 
~& ~ &hydrobase::y\_e(everywhere)\\ 
~& ~ &hydrobase::temperature(everywhere)\\ 
\end{tabular*} 


\vspace{5mm}

\noindent {\bf HydroBase\_Initial}   (conditional) 

\hspace{5mm} grhaylid\_constantdensitysphere 

\hspace{5mm}{\it set initial hydrodynamic data for 1d test } 


\hspace{5mm}

 \begin{tabular*}{160mm}{cll} 
~ & Language:  & c \\ 
~ & Reads:  & grid::coordinates \\ 
~ & Type:  & function \\ 
~ & Writes:  & hydrobase::rho(everywhere) \\ 
~& ~ &hydrobase::press(everywhere)\\ 
~& ~ &hydrobase::eps(everywhere)\\ 
~& ~ &hydrobase::vel(everywhere)\\ 
~& ~ &hydrobase::y\_e(everywhere)\\ 
~& ~ &hydrobase::temperature(everywhere)\\ 
\end{tabular*} 


\vspace{5mm}

\noindent {\bf CCTK\_INITIAL}   (conditional) 

\hspace{5mm} grhaylid\_betaequilibrium 

\hspace{5mm}{\it set y\_e, temperature, entropy, press, eps in neutrino free beta-equilibrium } 


\hspace{5mm}

 \begin{tabular*}{160mm}{cll} 
~ & After:  & hydrobase\_initial \\ 
~ & Before:  & hydrobase\_prim2coninitial \\ 
~ & Language:  & c \\ 
~ & Reads:  & hydrobase::rho \\ 
~ & Type:  & function \\ 
~ & Writes:  & hydrobase::press(everywhere) \\ 
~& ~ &hydrobase::eps(everywhere)\\ 
~& ~ &hydrobase::y\_e(everywhere)\\ 
~& ~ &hydrobase::temperature(everywhere)\\ 
\end{tabular*} 


\vspace{5mm}

\noindent {\bf CCTK\_INITIAL}   (conditional) 

\hspace{5mm} grhaylid\_compute\_entropy\_hybrid 

\hspace{5mm}{\it computes entropy from density and pressure using hybrid eos } 


\hspace{5mm}

 \begin{tabular*}{160mm}{cll} 
~ & After:  & hydrobase\_initial \\ 
~& ~ &impose\_beta\_equilibrium\\ 
~ & Before:  & hydrobase\_prim2coninitial \\ 
~ & Language:  & c \\ 
~ & Reads:  & hydrobase::rho \\ 
~& ~ &hydrobase::press\\ 
~ & Type:  & function \\ 
~ & Writes:  & hydrobase::entropy(everywhere) \\ 
\end{tabular*} 


\vspace{5mm}

\noindent {\bf CCTK\_INITIAL}   (conditional) 

\hspace{5mm} grhaylid\_compute\_entropy\_tabulated 

\hspace{5mm}{\it computes entropy from density, y\_e, and temperature using tabulated eos } 


\hspace{5mm}

 \begin{tabular*}{160mm}{cll} 
~ & After:  & hydrobase\_initial \\ 
~& ~ &impose\_beta\_equilibrium\\ 
~ & Before:  & hydrobase\_prim2coninitial \\ 
~ & Language:  & c \\ 
~ & Reads:  & hydrobase::rho \\ 
~& ~ &hydrobase::y\_e\\ 
~& ~ &hydrobase::temperature\\ 
~ & Type:  & function \\ 
~ & Writes:  & hydrobase::rho(everywhere) \\ 
~& ~ &hydrobase::press(everywhere)\\ 
~& ~ &hydrobase::eps(everywhere)\\ 
~& ~ &hydrobase::entropy(everywhere)\\ 
~& ~ &hydrobase::y\_e(everywhere)\\ 
~& ~ &hydrobase::temperature(everywhere)\\ 
\end{tabular*} 


\subsection*{Aliased Functions}

\hspace{5mm}

 \begin{tabular*}{160mm}{ll} 

{\bf Alias Name:} ~~~~~~~ & {\bf Function Name:} \\ 
GRHayLID\_BetaEquilibrium & impose\_beta\_equilibrium \\ 
\end{tabular*} 



\vspace{5mm}\parskip = 10pt 
\end{document}
