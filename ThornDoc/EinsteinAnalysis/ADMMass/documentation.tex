% *======================================================================*
%  Cactus Thorn template for ThornGuide documentation
%  Author: Ian Kelley
%  Date: Sun Jun 02, 2002
%  $Header$
%
%  Thorn documentation in the latex file doc/documentation.tex
%  will be included in ThornGuides built with the Cactus make system.
%  The scripts employed by the make system automatically include
%  pages about variables, parameters and scheduling parsed from the
%  relevant thorn CCL files.
%
%  This template contains guidelines which help to assure that your
%  documentation will be correctly added to ThornGuides. More
%  information is available in the Cactus UsersGuide.
%
%  Guidelines:
%   - Do not change anything before the line
%       % START CACTUS THORNGUIDE",
%     except for filling in the title, author, date, etc. fields.
%        - Each of these fields should only be on ONE line.
%        - Author names should be separated with a \\ or a comma.
%   - You can define your own macros, but they must appear after
%     the START CACTUS THORNGUIDE line, and must not redefine standard
%     latex commands.
%   - To avoid name clashes with other thorns, 'labels', 'citations',
%     'references', and 'image' names should conform to the following
%     convention:
%       ARRANGEMENT_THORN_LABEL
%     For example, an image wave.eps in the arrangement CactusWave and
%     thorn WaveToyC should be renamed to CactusWave_WaveToyC_wave.eps
%   - Graphics should only be included using the graphicx package.
%     More specifically, with the "\includegraphics" command.  Do
%     not specify any graphic file extensions in your .tex file. This
%     will allow us to create a PDF version of the ThornGuide
%     via pdflatex.
%   - References should be included with the latex "\bibitem" command.
%   - Use \begin{abstract}...\end{abstract} instead of \abstract{...}
%   - Do not use \appendix, instead include any appendices you need as
%     standard sections.
%   - For the benefit of our Perl scripts, and for future extensions,
%     please use simple latex.
%
% *======================================================================*
%
% Example of including a graphic image:
%    \begin{figure}[ht]
% 	\begin{center}
%    	   \includegraphics[width=6cm]{/home/runner/work/tests/tests/arrangements/EinsteinAnalysis/ADMMass/doc/MyArrangement_MyThorn_MyFigure}
% 	\end{center}
% 	\caption{Illustration of this and that}
% 	\label{MyArrangement_MyThorn_MyLabel}
%    \end{figure}
%
% Example of using a label:
%   \label{MyArrangement_MyThorn_MyLabel}
%
% Example of a citation:
%    \cite{MyArrangement_MyThorn_Author99}
%
% Example of including a reference
%   \bibitem{MyArrangement_MyThorn_Author99}
%   {J. Author, {\em The Title of the Book, Journal, or periodical}, 1 (1999),
%   1--16. {\tt http://www.nowhere.com/}}
%
% *======================================================================*

% If you are using CVS use this line to give version information
% $Header$

\documentclass{article}

% Use the Cactus ThornGuide style file
% (Automatically used from Cactus distribution, if you have a
%  thorn without the Cactus Flesh download this from the Cactus
%  homepage at www.cactuscode.org)
\usepackage{../../../../../doc/latex/cactus}

\newlength{\tableWidth} \newlength{\maxVarWidth} \newlength{\paraWidth} \newlength{\descWidth} \begin{document}

% The author of the documentation
\author{Frank Löffler}

% The title of the document (not necessarily the name of the Thorn)
\title{ADMMass}

% the date your document was last changed, if your document is in CVS,
% please use:
%    \date{$ $Date: 2004-01-07 14:12:39 -0600 (Wed, 07 Jan 2004) $ $}
\date{2012-05-07}

\maketitle

% Do not delete next line
% START CACTUS THORNGUIDE

% Add all definitions used in this documentation here
%   \def\mydef etc

% Add an abstract for this thorn's documentation
\begin{abstract}
Thorn ADMMass can compute the ADM mass from quantities in ADMBase.
\end{abstract}

% The following sections are suggestive only.
% Remove them or add your own.

\section{Introduction}
The ADM mass evaluated as either surface integrals at infinity or volume
integrals over entire hypersurfaces give a measure of the total energy in the
spacetime.

 The ADM mass can be defined~\cite{ADM_mass_Omurchadha74} as a surface integral over a
 sphere with infinite radius:
 \begin{equation}
 \label{eq:ADM_mass}
  M_{\mbox{\tiny ADM}}=\frac{1}{16\pi}\oint_\infty
  \sqrt{\gamma}\gamma^{ij}\gamma^{kl}
  (\gamma_{ik,j}-\gamma_{ij,k})\text{d}\! S_l
 \end{equation}.
 This is, assuming $\alpha=1$ at infinity, equivalent to
 \begin{equation}
 \label{eq:ADM_mass_volume}
  M_{\mbox{\tiny ADM}}=\frac{1}{16\pi}\int
  \left(\alpha\sqrt{\gamma}\gamma^{ij}\gamma^{kl}
  (\gamma_{ik,j}-\gamma_{ij,k})\right)_{,l}\text{d}\!^{\,3}\!\!x
 \end{equation}.
In practice, the following equation can also be used within the thorn:
  \begin{equation}
   \label{eq:ADM_mass_lapse}
   M_{\mbox{\tiny ADM}}=\frac{1}{16\pi}\oint_\infty
   \alpha\sqrt{\gamma}\gamma^{ij}\gamma^{kl}
   (\gamma_{ik,j}-\gamma_{ij,k})\text{d}\! S_l.
  \end{equation}
  This differs from equation~(\ref{eq:ADM_mass}) by the factor $\alpha$ inside
  the integral. For evaluations of those equations at infinity, $\alpha=1$ is
  assumed, and they are equal. For evaluations at a finite distance, however,
  this is usually not the case and the approximation of the ADM mass is
  gauge-dependent~\cite{ADM_mass_Omurchadha74}. Depending on circumstances,
  either~(\ref{eq:ADM_mass}),~(\ref{eq:ADM_mass_lapse} or~~(\ref{eq:ADM_mass_volume})
  might give better results.

\section{Using This Thorn}

Multiple measurements can be done for both volume and surface integral, but the
limit for both is 100 (change param.ccl if you need more). You need to
specify the number of integrations with \texttt{ADMMass\_number} (and ADMMass
will perform both integrations that many times).

Also note that this thorn uses the ADMMacros for derivatives. Thus, converegence
of results is limited to the order of these derivatives (ADMMacros::spatial\_order).

ADMMass provides several possibilities to specify the (finite) integration domain,
both for surface and volume integral, which we list in the following:

\begin{itemize}
 \item Surface Integral (over rectangular domain)
  \begin{itemize}
   \item \texttt{ADMMass\_distance\_from\_grid\_boundary}: specifies the distance
         between the physical domain boundary and the integration domain. If
         this is set, this fully specifies the domain boundary.
   \item \texttt{ADMMass\_surface\_distance}: specifies a distance of the integration
         boundary from a given point, specified using \texttt{ADMMass\_x\_pos[3]}.
   \item Otherwise, \texttt{ADMMass\_[xyz]\_[min|max]} specify a rectangular
         integration domain.
  \end{itemize}
 \item Volume Integral (over sphere)
  \begin{itemize}
   \item If \texttt{ADMMass\_use\_all\_volume\_as\_volume\_radius} is set,
         the whole volume is used for integration.
   \item If \texttt{ADMMass\_use\_surface\_distance\_as\_volume\_radius} is set
         and \texttt{ADMMass\_volume\_radius} is not (negative),
         \texttt{ADMMass\_surface\_distance} is used to specify the integration
         radius.
   \item Otherwise, \texttt{ADMMass\_volume\_radius} specifies this radius.
   \item \texttt{ADMMass\_[xyz]\_pos} specify the position of the integration
         sphere.
   \item Use \texttt{ADMMass\_Excise\_Horizons} to exclude domains where
         thorn \texttt{OutsideMask} didn't specify domain as \texttt{outside}.
         This can be used to, e.g. excise black hole apparent horizons.
  \end{itemize}
\end{itemize}

You should output \texttt{ADMMass::ADMMass\_Masses} for the result of the
integrations, which will include the results for the volume integral, the
usual surface integral and the sorface integral including the lapse.

\begin{thebibliography}{1}
\bibitem{ADM_mass_Omurchadha74}
N.~O Murchadha and J.~W.~York,
  ``Gravitational energy,''
  Phys.\ Rev.\ D {\bf 10}, 2345 (1974).
  doi:10.1103/PhysRevD.10.2345
  %%CITATION = doi:10.1103/PhysRevD.10.2345;%%  %%CITATION = doi:10.1103/PhysRevD.10.428;%%
\end{thebibliography}

% Do not delete next line
% END CACTUS THORNGUIDE




\section{Parameters} 


\parskip = 0pt

\setlength{\tableWidth}{160mm}

\setlength{\paraWidth}{\tableWidth}
\setlength{\descWidth}{\tableWidth}
\settowidth{\maxVarWidth}{admmass\_use\_surface\_distance\_as\_volume\_radius}

\addtolength{\paraWidth}{-\maxVarWidth}
\addtolength{\paraWidth}{-\columnsep}
\addtolength{\paraWidth}{-\columnsep}
\addtolength{\paraWidth}{-\columnsep}

\addtolength{\descWidth}{-\columnsep}
\addtolength{\descWidth}{-\columnsep}
\addtolength{\descWidth}{-\columnsep}
\noindent \begin{tabular*}{\tableWidth}{|c|l@{\extracolsep{\fill}}r|}
\hline
\multicolumn{1}{|p{\maxVarWidth}}{admmass\_debug} & {\bf Scope:} private & BOOLEAN \\\hline
\multicolumn{3}{|p{\descWidth}|}{{\bf Description:}   {\em Enable some info at runtime}} \\
\hline & & {\bf Default:} no \\\hline
\end{tabular*}

\vspace{0.5cm}\noindent \begin{tabular*}{\tableWidth}{|c|l@{\extracolsep{\fill}}r|}
\hline
\multicolumn{1}{|p{\maxVarWidth}}{admmass\_distance\_from\_grid\_boundary} & {\bf Scope:} private & REAL \\\hline
\multicolumn{3}{|p{\descWidth}|}{{\bf Description:}   {\em distance between the grid boundaries and the surface of the integration box}} \\
\hline{\bf Range} & &  {\bf Default:} -1.0 \\\multicolumn{1}{|p{\maxVarWidth}|}{\centering :} & \multicolumn{2}{p{\paraWidth}|}{{\textless}=0 for disable, positive otherwise} \\\hline
\end{tabular*}

\vspace{0.5cm}\noindent \begin{tabular*}{\tableWidth}{|c|l@{\extracolsep{\fill}}r|}
\hline
\multicolumn{1}{|p{\maxVarWidth}}{admmass\_excise\_horizons} & {\bf Scope:} private & BOOLEAN \\\hline
\multicolumn{3}{|p{\descWidth}|}{{\bf Description:}   {\em Should we exclude the region inside the AH to the volume integral}} \\
\hline & & {\bf Default:} no \\\hline
\end{tabular*}

\vspace{0.5cm}\noindent \begin{tabular*}{\tableWidth}{|c|l@{\extracolsep{\fill}}r|}
\hline
\multicolumn{1}{|p{\maxVarWidth}}{admmass\_number} & {\bf Scope:} private & INT \\\hline
\multicolumn{3}{|p{\descWidth}|}{{\bf Description:}   {\em number of measurements}} \\
\hline{\bf Range} & &  {\bf Default:} 1 \\\multicolumn{1}{|p{\maxVarWidth}|}{\centering 0:} & \multicolumn{2}{p{\paraWidth}|}{0 or positive} \\\hline
\end{tabular*}

\vspace{0.5cm}\noindent \begin{tabular*}{\tableWidth}{|c|l@{\extracolsep{\fill}}r|}
\hline
\multicolumn{1}{|p{\maxVarWidth}}{admmass\_surface\_distance} & {\bf Scope:} private & REAL \\\hline
\multicolumn{3}{|p{\descWidth}|}{{\bf Description:}   {\em distance between the above-defined center of the integration (cubic) box and its surface}} \\
\hline{\bf Range} & &  {\bf Default:} -1.0 \\\multicolumn{1}{|p{\maxVarWidth}|}{\centering :} & \multicolumn{2}{p{\paraWidth}|}{{\textless}=0 for disable, positive otherwise} \\\hline
\end{tabular*}

\vspace{0.5cm}\noindent \begin{tabular*}{\tableWidth}{|c|l@{\extracolsep{\fill}}r|}
\hline
\multicolumn{1}{|p{\maxVarWidth}}{admmass\_use\_all\_volume\_as\_volume\_radius} & {\bf Scope:} private & BOOLEAN \\\hline
\multicolumn{3}{|p{\descWidth}|}{{\bf Description:}   {\em Use the whole grid for volume integration}} \\
\hline & & {\bf Default:} no \\\hline
\end{tabular*}

\vspace{0.5cm}\noindent \begin{tabular*}{\tableWidth}{|c|l@{\extracolsep{\fill}}r|}
\hline
\multicolumn{1}{|p{\maxVarWidth}}{admmass\_use\_surface\_distance\_as\_volume\_radius} & {\bf Scope:} private & BOOLEAN \\\hline
\multicolumn{3}{|p{\descWidth}|}{{\bf Description:}   {\em Use ADMMass\_surface\_distance instead of ADMMass\_volume\_radius}} \\
\hline & & {\bf Default:} yes \\\hline
\end{tabular*}

\vspace{0.5cm}\noindent \begin{tabular*}{\tableWidth}{|c|l@{\extracolsep{\fill}}r|}
\hline
\multicolumn{1}{|p{\maxVarWidth}}{admmass\_volume\_radius} & {\bf Scope:} private & REAL \\\hline
\multicolumn{3}{|p{\descWidth}|}{{\bf Description:}   {\em radius of the sphere inside which the volume integral is computed}} \\
\hline{\bf Range} & &  {\bf Default:} -1.0 \\\multicolumn{1}{|p{\maxVarWidth}|}{\centering :} & \multicolumn{2}{p{\paraWidth}|}{{\textless}=0 for disable, positive otherwise} \\\hline
\end{tabular*}

\vspace{0.5cm}\noindent \begin{tabular*}{\tableWidth}{|c|l@{\extracolsep{\fill}}r|}
\hline
\multicolumn{1}{|p{\maxVarWidth}}{admmass\_x\_max} & {\bf Scope:} private & REAL \\\hline
\multicolumn{3}{|p{\descWidth}|}{{\bf Description:}   {\em x position of the righttmost yz-plane for the integration box}} \\
\hline{\bf Range} & &  {\bf Default:} 100.0 \\\multicolumn{1}{|p{\maxVarWidth}|}{\centering :} & \multicolumn{2}{p{\paraWidth}|}{anything} \\\hline
\end{tabular*}

\vspace{0.5cm}\noindent \begin{tabular*}{\tableWidth}{|c|l@{\extracolsep{\fill}}r|}
\hline
\multicolumn{1}{|p{\maxVarWidth}}{admmass\_x\_min} & {\bf Scope:} private & REAL \\\hline
\multicolumn{3}{|p{\descWidth}|}{{\bf Description:}   {\em x position of the leftmost yz-plane for the integration box}} \\
\hline{\bf Range} & &  {\bf Default:} -100.0 \\\multicolumn{1}{|p{\maxVarWidth}|}{\centering :} & \multicolumn{2}{p{\paraWidth}|}{anything} \\\hline
\end{tabular*}

\vspace{0.5cm}\noindent \begin{tabular*}{\tableWidth}{|c|l@{\extracolsep{\fill}}r|}
\hline
\multicolumn{1}{|p{\maxVarWidth}}{admmass\_x\_pos} & {\bf Scope:} private & REAL \\\hline
\multicolumn{3}{|p{\descWidth}|}{{\bf Description:}   {\em x position of the center of the integration box}} \\
\hline{\bf Range} & &  {\bf Default:} 0.0 \\\multicolumn{1}{|p{\maxVarWidth}|}{\centering :} & \multicolumn{2}{p{\paraWidth}|}{anything} \\\hline
\end{tabular*}

\vspace{0.5cm}\noindent \begin{tabular*}{\tableWidth}{|c|l@{\extracolsep{\fill}}r|}
\hline
\multicolumn{1}{|p{\maxVarWidth}}{admmass\_y\_max} & {\bf Scope:} private & REAL \\\hline
\multicolumn{3}{|p{\descWidth}|}{{\bf Description:}   {\em y position of the rightmost xz-plane for the integration box}} \\
\hline{\bf Range} & &  {\bf Default:} 100.0 \\\multicolumn{1}{|p{\maxVarWidth}|}{\centering :} & \multicolumn{2}{p{\paraWidth}|}{anything} \\\hline
\end{tabular*}

\vspace{0.5cm}\noindent \begin{tabular*}{\tableWidth}{|c|l@{\extracolsep{\fill}}r|}
\hline
\multicolumn{1}{|p{\maxVarWidth}}{admmass\_y\_min} & {\bf Scope:} private & REAL \\\hline
\multicolumn{3}{|p{\descWidth}|}{{\bf Description:}   {\em y position of the leftmost xz-plane for the integration box}} \\
\hline{\bf Range} & &  {\bf Default:} -100.0 \\\multicolumn{1}{|p{\maxVarWidth}|}{\centering :} & \multicolumn{2}{p{\paraWidth}|}{anything} \\\hline
\end{tabular*}

\vspace{0.5cm}\noindent \begin{tabular*}{\tableWidth}{|c|l@{\extracolsep{\fill}}r|}
\hline
\multicolumn{1}{|p{\maxVarWidth}}{admmass\_y\_pos} & {\bf Scope:} private & REAL \\\hline
\multicolumn{3}{|p{\descWidth}|}{{\bf Description:}   {\em y position of the center of the integration box}} \\
\hline{\bf Range} & &  {\bf Default:} 0.0 \\\multicolumn{1}{|p{\maxVarWidth}|}{\centering :} & \multicolumn{2}{p{\paraWidth}|}{anything} \\\hline
\end{tabular*}

\vspace{0.5cm}\noindent \begin{tabular*}{\tableWidth}{|c|l@{\extracolsep{\fill}}r|}
\hline
\multicolumn{1}{|p{\maxVarWidth}}{admmass\_z\_max} & {\bf Scope:} private & REAL \\\hline
\multicolumn{3}{|p{\descWidth}|}{{\bf Description:}   {\em z position of the rightmost xy-plane for the integration box}} \\
\hline{\bf Range} & &  {\bf Default:} 100.0 \\\multicolumn{1}{|p{\maxVarWidth}|}{\centering :} & \multicolumn{2}{p{\paraWidth}|}{anything} \\\hline
\end{tabular*}

\vspace{0.5cm}\noindent \begin{tabular*}{\tableWidth}{|c|l@{\extracolsep{\fill}}r|}
\hline
\multicolumn{1}{|p{\maxVarWidth}}{admmass\_z\_min} & {\bf Scope:} private & REAL \\\hline
\multicolumn{3}{|p{\descWidth}|}{{\bf Description:}   {\em z position of the leftmost xy-plane for the integration box}} \\
\hline{\bf Range} & &  {\bf Default:} -100.0 \\\multicolumn{1}{|p{\maxVarWidth}|}{\centering :} & \multicolumn{2}{p{\paraWidth}|}{anything} \\\hline
\end{tabular*}

\vspace{0.5cm}\noindent \begin{tabular*}{\tableWidth}{|c|l@{\extracolsep{\fill}}r|}
\hline
\multicolumn{1}{|p{\maxVarWidth}}{admmass\_z\_pos} & {\bf Scope:} private & REAL \\\hline
\multicolumn{3}{|p{\descWidth}|}{{\bf Description:}   {\em z position of the center of the integration box}} \\
\hline{\bf Range} & &  {\bf Default:} 0.0 \\\multicolumn{1}{|p{\maxVarWidth}|}{\centering :} & \multicolumn{2}{p{\paraWidth}|}{anything} \\\hline
\end{tabular*}

\vspace{0.5cm}\parskip = 10pt 

\section{Interfaces} 


\parskip = 0pt

\vspace{3mm} \subsection*{General}

\noindent {\bf Implements}: 

admmass
\vspace{2mm}

\noindent {\bf Inherits}: 

admbase

admmacros

staticconformal

spacemask
\vspace{2mm}
\subsection*{Grid Variables}
\vspace{5mm}\subsubsection{PRIVATE GROUPS}

\vspace{5mm}

\begin{tabular*}{150mm}{|c|c@{\extracolsep{\fill}}|rl|} \hline 
~ {\bf Group Names} ~ & ~ {\bf Variable Names} ~  &{\bf Details} ~ & ~\\ 
\hline 
admmass\_loopcounterg &  & compact & 0 \\ 
 & ADMMass\_LoopCounter & description & ADMMass LoopCounter \\ 
 &  & dimensions & 0 \\ 
 &  & distribution & CONSTANT \\ 
 &  & group type & SCALAR \\ 
 &  & timelevels & 1 \\ 
 &  & variable type & INT \\ 
\hline 
admmass\_masses &  & compact & 0 \\ 
 & ADMMass\_SurfaceMass & description & ADMMass Scalars \\ 
 & ADMMass\_SurfaceMass\_Lapse & dimensions & 0 \\ 
 & ADMMass\_VolumeMass & distribution & CONSTANT \\ 
 &  & group type & SCALAR \\ 
 &  & tags & checkpoint="no" \\ 
 &  & timelevels & 1 \\ 
 &  & vararray\_size & ADMMass\_number \\ 
 &  & variable type & REAL \\ 
\hline 
admmass\_gfs\_surface &  & compact & 0 \\ 
 & ADMMass\_SurfaceMass\_GF & description & ADMMass gridfunctions for surface integration \\ 
 &  & dimensions & 3 \\ 
 &  & distribution & DEFAULT \\ 
 &  & group type & GF \\ 
 &  & tags & Prolongation="none" tensortypealias="Scalar" checkpoint="no" \\ 
 &  & timelevels & 3 \\ 
 &  & variable type & REAL \\ 
\hline 
admmass\_gfs\_volume &  & compact & 0 \\ 
 & ADMMass\_VolumeMass\_pot\_x & description & ADMMass gridfunctions for volume integration \\ 
 & ADMMass\_VolumeMass\_pot\_y & dimensions & 3 \\ 
 & ADMMass\_VolumeMass\_pot\_z & distribution & DEFAULT \\ 
 & ADMMass\_VolumeMass\_GF & group type & GF \\ 
 &  & tags & Prolongation="none" tensortypealias="Scalar" checkpoint="no" \\ 
 &  & timelevels & 3 \\ 
 &  & variable type & REAL \\ 
\hline 
admmass\_box &  & compact & 0 \\ 
 & ADMMass\_box\_x\_min & description & Physical coordinates of the surface on which the integral is computed \\ 
 & ADMMass\_box\_x\_max & dimensions & 0 \\ 
 & ADMMass\_box\_y\_min & distribution & CONSTANT \\ 
 & ADMMass\_box\_y\_max & group type & SCALAR \\ 
 & ADMMass\_box\_z\_min & tags & checkpoint="no" \\ 
 & ADMMass\_box\_z\_max & timelevels & 1 \\ 
 &  & variable type & REAL \\ 
\hline 
grid\_spacing\_product & grid\_spacing\_product & compact & 0 \\ 
 &  & description & product of cctk\_delta\_space \\ 
& ~ & description &  to be computed in local mode and later used in global mode (carpet problems) \\ 
 &  & dimensions & 0 \\ 
 &  & distribution & CONSTANT \\ 
 &  & group type & SCALAR \\ 
 &  & tags & checkpoint="no" \\ 
 &  & timelevels & 1 \\ 
 &  & variable type & REAL \\ 
\hline 
\end{tabular*} 



\vspace{5mm}

\noindent {\bf Uses header}: 

SpaceMask.h
\vspace{2mm}\parskip = 10pt 

\section{Schedule} 


\parskip = 0pt


\noindent This section lists all the variables which are assigned storage by thorn EinsteinAnalysis/ADMMass.  Storage can either last for the duration of the run ({\bf Always} means that if this thorn is activated storage will be assigned, {\bf Conditional} means that if this thorn is activated storage will be assigned for the duration of the run if some condition is met), or can be turned on for the duration of a schedule function.


\subsection*{Storage}

\hspace{5mm}

 \begin{tabular*}{160mm}{ll} 

{\bf Always:}&  ~ \\ 
 ADMMass\_LoopCounterG & ~\\ 
 ADMMass\_Masses & ~\\ 
 ADMMass\_GFs\_surface[3] & ~\\ 
 ADMMass\_GFs\_volume[3] & ~\\ 
~ & ~\\ 
\end{tabular*} 


\subsection*{Scheduled Functions}
\vspace{5mm}

\noindent {\bf CCTK\_INITIAL} 

\hspace{5mm} admmass\_initloopcounter 

\hspace{5mm}{\it initialise the loop counter for admmass } 


\hspace{5mm}

 \begin{tabular*}{160mm}{cll} 
~ & Language:  & c \\ 
~ & Options:  & global \\ 
~ & Type:  & function \\ 
\end{tabular*} 


\vspace{5mm}

\noindent {\bf CCTK\_POSTSTEP} 

\hspace{5mm} admmass\_setloopcounter 

\hspace{5mm}{\it set the loop counter to the value of the parameter admmass:admmass\_number } 


\hspace{5mm}

 \begin{tabular*}{160mm}{cll} 
~ & After:  & outsidemask\_updatemask \\ 
~ & Language:  & c \\ 
~ & Options:  & global \\ 
~ & Type:  & function \\ 
\end{tabular*} 


\vspace{5mm}

\noindent {\bf CCTK\_POSTSTEP} 

\hspace{5mm} admmass 

\hspace{5mm}{\it admmass loop } 


\hspace{5mm}

 \begin{tabular*}{160mm}{cll} 
~ & After:  & admmass\_setloopcounter \\ 
~ & Storage:  & admmass\_gfs\_surface[3] \\ 
~& ~ &admmass\_gfs\_volume[3]\\ 
~& ~ &admmass\_box\\ 
~& ~ &grid\_spacing\_product\\ 
~ & Type:  & group \\ 
~ & While:  & admmass::admmass\_loopcounter \\ 
\end{tabular*} 


\vspace{5mm}

\noindent {\bf ADMMass} 

\hspace{5mm} admmass\_loop 

\hspace{5mm}{\it decrement loop counter } 


\hspace{5mm}

 \begin{tabular*}{160mm}{cll} 
~ & Language:  & c \\ 
~ & Options:  & global \\ 
~ & Type:  & function \\ 
\end{tabular*} 


\vspace{5mm}

\noindent {\bf ADMMass} 

\hspace{5mm} admmass\_surface 

\hspace{5mm}{\it calculate the admmass using a surface integral: local routine } 


\hspace{5mm}

 \begin{tabular*}{160mm}{cll} 
~ & After:  & admmass\_loop \\ 
~ & Language:  & c \\ 
~ & Options:  & global \\ 
~& ~ &loop-local\\ 
~ & Sync:  & admmass\_gfs\_surface \\ 
~ & Type:  & function \\ 
\end{tabular*} 


\vspace{5mm}

\noindent {\bf ADMMass} 

\hspace{5mm} admmass\_surface\_global 

\hspace{5mm}{\it calculate the admmass using a surface integral: global routine } 


\hspace{5mm}

 \begin{tabular*}{160mm}{cll} 
~ & After:  & admmass\_surface \\ 
~ & Language:  & c \\ 
~ & Options:  & global \\ 
~ & Type:  & function \\ 
\end{tabular*} 


\vspace{5mm}

\noindent {\bf ADMMass} 

\hspace{5mm} admmass\_surface\_lapse 

\hspace{5mm}{\it calculate the admmass*lapse using a surface integral: local routine } 


\hspace{5mm}

 \begin{tabular*}{160mm}{cll} 
~ & After:  & admmass\_surface\_global \\ 
~ & Language:  & c \\ 
~ & Options:  & global \\ 
~& ~ &loop-local\\ 
~ & Sync:  & admmass\_gfs\_surface \\ 
~ & Type:  & function \\ 
\end{tabular*} 


\vspace{5mm}

\noindent {\bf ADMMass} 

\hspace{5mm} admmass\_surface\_lapse\_global 

\hspace{5mm}{\it calculate the admmass*lapse using a surface integral: global routine } 


\hspace{5mm}

 \begin{tabular*}{160mm}{cll} 
~ & After:  & admmass\_surface\_lapse \\ 
~ & Language:  & c \\ 
~ & Options:  & global \\ 
~ & Type:  & function \\ 
\end{tabular*} 


\vspace{5mm}

\noindent {\bf ADMMass} 

\hspace{5mm} admmass\_volume 

\hspace{5mm}{\it calculate the admmass using a volume integral: local routine } 


\hspace{5mm}

 \begin{tabular*}{160mm}{cll} 
~ & After:  & admmass\_surface\_lapse\_global \\ 
~ & Language:  & c \\ 
~ & Options:  & global \\ 
~& ~ &loop-local\\ 
~ & Sync:  & admmass\_gfs\_volume \\ 
~ & Type:  & function \\ 
\end{tabular*} 


\vspace{5mm}

\noindent {\bf ADMMass} 

\hspace{5mm} admmass\_volume\_global 

\hspace{5mm}{\it calculate the admmass using a volume integral: global routine } 


\hspace{5mm}

 \begin{tabular*}{160mm}{cll} 
~ & After:  & admmass\_volume \\ 
~ & Language:  & c \\ 
~ & Options:  & global \\ 
~ & Type:  & function \\ 
\end{tabular*} 



\vspace{5mm}\parskip = 10pt 
\end{document}
