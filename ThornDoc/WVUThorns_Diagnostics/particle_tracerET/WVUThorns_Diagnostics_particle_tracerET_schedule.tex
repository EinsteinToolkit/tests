
\section{Schedule} 


\parskip = 0pt


\noindent This section lists all the variables which are assigned storage by thorn WVUThorns\_Diagnostics/particle\_tracerET.  Storage can either last for the duration of the run ({\bf Always} means that if this thorn is activated storage will be assigned, {\bf Conditional} means that if this thorn is activated storage will be assigned for the duration of the run if some condition is met), or can be turned on for the duration of a schedule function.


\subsection*{Storage}

\hspace{5mm}

 \begin{tabular*}{160mm}{ll} 

{\bf Always:}&  ~ \\ 
 RK4IterationCounterVar particle\_position\_arrays velocity\_consistent\_with\_MHD\_induction\_equations & ~\\ 
 ADMBase::metric[metric\_timelevels] ADMBase::curv[metric\_timelevels] ADMBase::lapse[lapse\_timelevels] ADMBase::shift[shift\_timelevels] & ~\\ 
 HydroBase::rho[timelevels] HydroBase::press[timelevels] HydroBase::eps[timelevels] HydroBase::vel[timelevels] HydroBase::Bvec[timelevels] & ~\\ 
~ & ~\\ 
\end{tabular*} 


\subsection*{Scheduled Functions}
\vspace{5mm}

\noindent {\bf CCTK\_INITIAL} 

\hspace{5mm} particle\_traceret\_file\_output\_routine\_startup 

\hspace{5mm}{\it create directory for file output. } 


\hspace{5mm}

 \begin{tabular*}{160mm}{cll} 
~ & Language:  & c \\ 
~ & Type:  & function \\ 
\end{tabular*} 


\vspace{5mm}

\noindent {\bf CCTK\_ANALYSIS} 

\hspace{5mm} convert\_to\_mhd\_3velocity 

\hspace{5mm}{\it convert to mhd 3 velocity } 


\hspace{5mm}

 \begin{tabular*}{160mm}{cll} 
~ & Before:  & particle\_traceret \\ 
~ & Language:  & c \\ 
~ & Options:  & global-early \\ 
~& ~ &loop-local\\ 
~ & Type:  & function \\ 
\end{tabular*} 


\vspace{5mm}

\noindent {\bf CCTK\_ANALYSIS} 

\hspace{5mm} particle\_traceret 

\hspace{5mm}{\it particle traceret subroutines } 


\hspace{5mm}

 \begin{tabular*}{160mm}{cll} 
~ & Type:  & group \\ 
\end{tabular*} 


\vspace{5mm}

\noindent {\bf particle\_tracerET} 

\hspace{5mm} initializerk4iterationcountertoone 

\hspace{5mm}{\it initialize rk4iterationcounter variable to one } 


\hspace{5mm}

 \begin{tabular*}{160mm}{cll} 
~ & Language:  & c \\ 
~ & Options:  & global-early \\ 
~ & Type:  & function \\ 
\end{tabular*} 


\vspace{5mm}

\noindent {\bf particle\_tracerET} 

\hspace{5mm} initializeparticlepositions 

\hspace{5mm}{\it initialize particle positions } 


\hspace{5mm}

 \begin{tabular*}{160mm}{cll} 
~ & After:  & initializerk4iterationcountertoone \\ 
~ & Language:  & c \\ 
~ & Options:  & global-early \\ 
~ & Type:  & function \\ 
\end{tabular*} 


\vspace{5mm}

\noindent {\bf particle\_tracerET} 

\hspace{5mm} doonerk4stepforparticletraceret 

\hspace{5mm}{\it do one rk4 step for particle traceret } 


\hspace{5mm}

 \begin{tabular*}{160mm}{cll} 
~ & After:  & initializeparticlepositions \\ 
~ & Language:  & c \\ 
~ & Options:  & global-early \\ 
~ & Type:  & function \\ 
\end{tabular*} 


\vspace{5mm}

\noindent {\bf particle\_tracerET} 

\hspace{5mm} particle\_traceret\_file\_output 

\hspace{5mm}{\it output particle traceret data to disk } 


\hspace{5mm}

 \begin{tabular*}{160mm}{cll} 
~ & After:  & doonerk4stepforparticletraceret \\ 
~ & Language:  & c \\ 
~ & Options:  & global-early \\ 
~ & Type:  & function \\ 
\end{tabular*} 



\vspace{5mm}\parskip = 10pt 
