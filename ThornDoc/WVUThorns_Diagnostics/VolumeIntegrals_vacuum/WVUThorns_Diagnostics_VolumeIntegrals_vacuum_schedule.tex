
\section{Schedule} 


\parskip = 0pt


\noindent This section lists all the variables which are assigned storage by thorn WVUThorns\_Diagnostics/VolumeIntegrals\_vacuum.  Storage can either last for the duration of the run ({\bf Always} means that if this thorn is activated storage will be assigned, {\bf Conditional} means that if this thorn is activated storage will be assigned for the duration of the run if some condition is met), or can be turned on for the duration of a schedule function.


\subsection*{Storage}

\hspace{5mm}

 \begin{tabular*}{160mm}{ll} 

{\bf Always:}&  ~ \\ 
 VolIntegrands VolIntegrals MovingSphRegionIntegrals IntegralCounterVar VolIntegrals\_vacuum\_time & ~\\ 
~ & ~\\ 
\end{tabular*} 


\subsection*{Scheduled Functions}
\vspace{5mm}

\noindent {\bf CCTK\_INITIAL}   (conditional) 

\hspace{5mm} vi\_vacuum\_file\_output\_routine\_startup 

\hspace{5mm}{\it create directory for file output. } 


\hspace{5mm}

 \begin{tabular*}{160mm}{cll} 
~ & Language:  & c \\ 
~ & Type:  & function \\ 
\end{tabular*} 


\vspace{5mm}

\noindent {\bf CCTK\_POST\_RECOVER\_VARIABLES}   (conditional) 

\hspace{5mm} vi\_vacuum\_file\_output\_routine\_startup 

\hspace{5mm}{\it create directory for file output. } 


\hspace{5mm}

 \begin{tabular*}{160mm}{cll} 
~ & Language:  & c \\ 
~ & Type:  & function \\ 
\end{tabular*} 


\vspace{5mm}

\noindent {\bf CCTK\_INITIAL} 

\hspace{5mm} initializeintegralcountertozero 

\hspace{5mm}{\it initialize integralcounter variable to zero } 


\hspace{5mm}

 \begin{tabular*}{160mm}{cll} 
~ & Language:  & c \\ 
~ & Options:  & global \\ 
~ & Type:  & function \\ 
\end{tabular*} 


\vspace{5mm}

\noindent {\bf CCTK\_POST\_RECOVER\_VARIABLES} 

\hspace{5mm} initializeintegralcountertozero 

\hspace{5mm}{\it initialize integralcounter variable to zero } 


\hspace{5mm}

 \begin{tabular*}{160mm}{cll} 
~ & Language:  & c \\ 
~ & Options:  & global \\ 
~ & Type:  & function \\ 
\end{tabular*} 


\vspace{5mm}

\noindent {\bf CCTK\_ANALYSIS} 

\hspace{5mm} initializeintegralcounter 

\hspace{5mm}{\it initialize integralcounter variable } 


\hspace{5mm}

 \begin{tabular*}{160mm}{cll} 
~ & Before:  & volumeintegralgroup \\ 
~ & Language:  & c \\ 
~ & Options:  & global \\ 
~ & Type:  & function \\ 
\end{tabular*} 


\vspace{5mm}

\noindent {\bf CCTK\_ANALYSIS} 

\hspace{5mm} volumeintegralgroup 

\hspace{5mm}{\it evaluate all volume integrals } 


\hspace{5mm}

 \begin{tabular*}{160mm}{cll} 
~ & Before:  & carpetlib\_printtimestats \\ 
~& ~ &carpetlib\_printmemstats\\ 
~ & Type:  & group \\ 
~ & While:  & volumeintegrals\_vacuum::integralcounter \\ 
\end{tabular*} 


\vspace{5mm}

\noindent {\bf VolumeIntegralGroup} 

\hspace{5mm} volumeintegrals\_vacuum\_computeintegrand 

\hspace{5mm}{\it compute integrand } 


\hspace{5mm}

 \begin{tabular*}{160mm}{cll} 
~ & Before:  & dosum \\ 
~ & Language:  & c \\ 
~ & Options:  & global \\ 
~& ~ &loop-local\\ 
~ & Storage:  & volintegrands \\ 
~& ~ &volintegrals\\ 
~& ~ &movingsphregionintegrals\\ 
~ & Type:  & function \\ 
\end{tabular*} 


\vspace{5mm}

\noindent {\bf VolumeIntegralGroup} 

\hspace{5mm} dosum 

\hspace{5mm}{\it do sum } 


\hspace{5mm}

 \begin{tabular*}{160mm}{cll} 
~ & After:  & volumeintegrals\_vacuum\_computeintegrand \\ 
~ & Language:  & c \\ 
~ & Options:  & global \\ 
~ & Type:  & function \\ 
\end{tabular*} 


\vspace{5mm}

\noindent {\bf VolumeIntegralGroup} 

\hspace{5mm} decrementintegralcounter 

\hspace{5mm}{\it decrement integralcounter variable } 


\hspace{5mm}

 \begin{tabular*}{160mm}{cll} 
~ & After:  & dosum \\ 
~ & Language:  & c \\ 
~ & Options:  & global \\ 
~ & Type:  & function \\ 
\end{tabular*} 


\vspace{5mm}

\noindent {\bf CCTK\_ANALYSIS} 

\hspace{5mm} vi\_vacuum\_file\_output 

\hspace{5mm}{\it output volumeintegral results to disk } 


\hspace{5mm}

 \begin{tabular*}{160mm}{cll} 
~ & After:  & volumeintegralgroup \\ 
~ & Language:  & c \\ 
~ & Options:  & global \\ 
~ & Type:  & function \\ 
\end{tabular*} 



\vspace{5mm}\parskip = 10pt 
