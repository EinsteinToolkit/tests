
\section{Interfaces} 


\parskip = 0pt

\vspace{3mm} \subsection*{General}

\noindent {\bf Implements}: 

volumeintegrals\_vacuum
\vspace{2mm}

\noindent {\bf Inherits}: 

grid

admbase

carpetregrid2
\vspace{2mm}
\subsection*{Grid Variables}
\vspace{5mm}\subsubsection{PRIVATE GROUPS}

\vspace{5mm}

\begin{tabular*}{150mm}{|c|c@{\extracolsep{\fill}}|rl|} \hline 
~ {\bf Group Names} ~ & ~ {\bf Variable Names} ~  &{\bf Details} ~ & ~\\ 
\hline 
volintegrands &  & compact & 0 \\ 
 & VolIntegrand1 & dimensions & 3 \\ 
 & VolIntegrand2 & distribution & DEFAULT \\ 
 & VolIntegrand3 & group type & GF \\ 
 & VolIntegrand4 & tags & InterpNumTimelevels=1 prolongation="none" Checkpoint="no" \\ 
 &  & timelevels & 1 \\ 
 &  & variable type & REAL \\ 
\hline 
volintegrals &  & compact & 0 \\ 
 & VolIntegral & description & Volume integrals \\ 
& ~ & description &  post-sum. The first dimension denotes which integral(s) \\ 
 & VolIntegral & description &  and the second denotes the values of the integral(s). E.g. \\ 
 & VolIntegral & description &  a center of mass volume integral will have 3 outputs. \\ 
 &  & dimensions & 2 \\ 
 &  & distribution & CONSTANT \\ 
 &  & group type & ARRAY \\ 
 &  & size & 101 \\ 
& ~ & size & 4 \\ 
 &  & timelevels & 1 \\ 
 &  & variable type & REAL \\ 
\hline 
movingsphregionintegrals &  & compact & 0 \\ 
 & volintegral\_inside\_sphere\_\_center\_x & description & Specify regions for volume integrals inside/outside spheres THAT MOVE. \\ 
 & volintegral\_inside\_sphere\_\_center\_y & dimensions & 1 \\ 
 & volintegral\_inside\_sphere\_\_center\_z & distribution & CONSTANT \\ 
 & volintegral\_outside\_sphere\_\_center\_x & group type & ARRAY \\ 
 & volintegral\_outside\_sphere\_\_center\_y & size & 101 \\ 
 & volintegral\_outside\_sphere\_\_center\_z & timelevels & 1 \\ 
 &  & variable type & REAL \\ 
\hline 
\end{tabular*} 


\vspace{5mm}\subsubsection{PUBLIC GROUPS}

\vspace{5mm}

\begin{tabular*}{150mm}{|c|c@{\extracolsep{\fill}}|rl|} \hline 
~ {\bf Group Names} ~ & ~ {\bf Variable Names} ~  &{\bf Details} ~ & ~\\ 
\hline 
integralcountervar &  & compact & 0 \\ 
 & IntegralCounter & description & Counter that keeps track of which integral we are calculating. \\ 
 &  & dimensions & 0 \\ 
 &  & distribution & CONSTANT \\ 
 &  & group type & SCALAR \\ 
 &  & tags & checkpoint="no" \\ 
 &  & timelevels & 1 \\ 
 &  & variable type & INT \\ 
\hline 
volintegrals\_vacuum\_time &  & compact & 0 \\ 
 & physical\_time & description & keeps track of the physical time \\ 
& ~ & description &  in case time coordinate is reparameterized \\ 
 & physical\_time & description &  a la http://arxiv.org/abs/1404.6523 \\ 
 &  & dimensions & 0 \\ 
 &  & distribution & CONSTANT \\ 
 &  & group type & SCALAR \\ 
 &  & tags & checkpoint="no" \\ 
 &  & timelevels & 1 \\ 
 &  & variable type & REAL \\ 
\hline 
\end{tabular*} 



\vspace{5mm}\parskip = 10pt 
