% *======================================================================*
%  Cactus Thorn template for ThornGuide documentation
%  Author: Ian Kelley
%  Date: Sun Jun 02, 2002
%
%  Thorn documentation in the latex file doc/documentation.tex 
%  will be included in ThornGuides built with the Cactus make system.
%  The scripts employed by the make system automatically include 
%  pages about variables, parameters and scheduling parsed from the 
%  relevant thorn CCL files.
%  
%  This template contains guidelines which help to assure that your     
%  documentation will be correctly added to ThornGuides. More 
%  information is available in the Cactus UsersGuide.
%                                                    
%  Guidelines:
%   - Do not change anything before the line
%       % START CACTUS THORNGUIDE",
%     except for filling in the title, author, date, etc. fields.
%        - Each of these fields should only be on ONE line.
%        - Author names should be separated with a \\ or a comma.
%   - You can define your own macros, but they must appear after
%     the START CACTUS THORNGUIDE line, and must not redefine standard 
%     latex commands.
%   - To avoid name clashes with other thorns, 'labels', 'citations', 
%     'references', and 'image' names should conform to the following 
%     convention:          
%       ARRANGEMENT_THORN_LABEL
%     For example, an image wave.eps in the arrangement CactusWave and 
%     thorn WaveToyC should be renamed to CactusWave_WaveToyC_wave.eps
%   - Graphics should only be included using the graphicx package. 
%     More specifically, with the "\includegraphics" command.  Do
%     not specify any graphic file extensions in your .tex file. This 
%     will allow us to create a PDF version of the ThornGuide
%     via pdflatex.
%   - References should be included with the latex "\bibitem" command. 
%   - Use \begin{abstract}...\end{abstract} instead of \abstract{...}
%   - Do not use \appendix, instead include any appendices you need as 
%     standard sections. 
%   - For the benefit of our Perl scripts, and for future extensions, 
%     please use simple latex.     
%
% *======================================================================* 
% 
% Example of including a graphic image:
%    \begin{figure}[ht]
% 	\begin{center}
%    	   \includegraphics[width=6cm]{/home/runner/work/tests/tests/arrangements/CactusNumerical/SphericalSurface/doc/MyArrangement_MyThorn_MyFigure}
% 	\end{center}
% 	\caption{Illustration of this and that}
% 	\label{MyArrangement_MyThorn_MyLabel}
%    \end{figure}
%
% Example of using a label:
%   \label{MyArrangement_MyThorn_MyLabel}
%
% Example of a citation:
%    \cite{MyArrangement_MyThorn_Author99}
%
% Example of including a reference
%   \bibitem{MyArrangement_MyThorn_Author99}
%   {J. Author, {\em The Title of the Book, Journal, or periodical}, 1 (1999), 
%   1--16. {\tt http://www.nowhere.com/}}
%
% *======================================================================* 

\documentclass{article}

% Use the Cactus ThornGuide style file
% (Automatically used from Cactus distribution, if you have a 
%  thorn without the Cactus Flesh download this from the Cactus
%  homepage at www.cactuscode.org)
\usepackage{../../../../../doc/latex/cactus}

\newlength{\tableWidth} \newlength{\maxVarWidth} \newlength{\paraWidth} \newlength{\descWidth} \begin{document}

% The author of the documentation
\author{Erik Schnetter \textless schnetter@cct.lsu.edu\textgreater}

% The title of the document (not necessarily the name of the Thorn)
\title{SphericalSurface}

% the date your document was last changed
\date{2007-03-06}

\maketitle

% Do not delete next line
% START CACTUS THORNGUIDE

% Add all definitions used in this documentation here 
%   \def\mydef etc

\begin{abstract}
This thorn provides a repository for two-dimensional surfaces with
spherical topology.  This thorn does not actually do anything with
these surfaces, but allows other thorns to store and retrieve the
surfaces and some associated information.  The exact interpretation of
the stored quantities is up to the using thorns, but certain standard
definitions are suggested.
\end{abstract}



\section{Introduction}

Many thorns work on manifolds that are two-dimensional, closed
surfaces.  Examples are apparent and event horizons, or the surfaces
on which gravitational waves are extracted.  Other such surfaces might
be excision or outer boundaries (although these are currently not
treated as such).  There is a need to have a common representation for
such surfaces, so that the surface-finding thorns and the thorns
working with these surfaces can be independent.  A common
representation will also facilitate visualisation.  This thorn
\texttt{SphericalSurface} provides just such a common representation.

This thorn is not meant to do anything else but be a ``repository''
for surfaces.  It is up to the surface-finding and surface-using
thorns to agree on the details of the information stored.  Of course,
standard definitions for the stored quantities are suggested.  (For
example, there is no exact definition of the quadrupole moment,
because this definition will depend on the kind of surface that is
stored.  However, it is specified that the quadrupole moment should be
calculate with respect to the centroid, and that it should not be
trace-free.)

This thorn provides storage for several independent surfaces,
identified by an \emph{index}.  It is up to the user to specify,
probably in the parameter file, which thorns use what surfaces for
what purpose.



\section{Surface Definition}

This thorn provides, for each surface, a two-dimensional grid array
\code{sf\_radius} and grid scalars \code{sf\_origin\_x},
\code{sf\_origin\_y}, and \code{sf\_origin\_z}.  The number of
surfaces is determined by the parameter \code{nsurfaces}, which has
to be set in the parameter file.

\code{sf\_radius} should contain the radius of the surface as
measured from its origin, where the arrays indices vary in the
$\theta$ and $\phi$ direction, respectively.  Both the radius array
and the surface origin are supposed to be set when a surface is
stored.

The coordinates on the surface, i.e.\ the grid origin and spacing in
the $\theta$ and $\phi$ directions, is available from the grid scalars
\code{sf\_origin\_theta}, \code{sf\_origin\_phi},
\code{sf\_delta\_theta}, and \code{sf\_delta\_phi}.  These grid
scalars are set by the thorn SphericalSurface in the \code{basegrid}
bin, and are meant to be read-only for other thorns.



\begin{quotation}
\subsection*{A note on vector grid variables}

A relatively new addition to Cactus (in November 2003) are vector grid
variables.  These are essentially arrays of grid variables.  Thorn
SphericalSurface makes use of these by storing the surfaces in such
arrays.  That means that in order to access data from a single
surface, on has to use the corresponding surface index as array index.
In a similar manner, thorn SurfaceIndex uses array parameters for its
parameters (except certain global ones).

This should be kept in mind when writing source code.  C has the
unfortunate property of converting arrays into meaningless integers if
an array subscript is accidentally omitted.  Fortran knows whole-array
expressions, meaning that it would act on all surfaces instead of a
single one if an array subscript is accidentally omitted.

Each element of a vector grid function is a grid function.  (The term
``grid function vector'' might have been more appropriate.)  As such,
it has a name, which can be used e.g.\ for output.  The name consists
of the vector grid function name to which the surface index in square
brackets has been appended.
\end{quotation}



\subsection{Global Surface Quantities}

In many cases, only some abstract information about the surface is of
interest, such as its mean radius or its quadrupole moment.  For that
purpose there are additional grid scalars that carry this information.
These grid scalars are also supposed to be set when a surface is
stored.  These grid scalars are
\begin{description}

\item[\code{sf\_mean\_radius}] Mean of the surface radius.  This
should be the arithmetic mean where the radii have been weighted with
$\sin\theta$, or a suitable generalisation thereof.  This quantity is
also supposed to be a measure of the surface's monopole moment.  One
suggested expression is $M=\sqrt{A}$ with $A = \int_S d\Omega\, r^2 /
\int_S d\Omega$.

\item[\code{sf\_min\_radius}, \code{sf\_max\_radius}] Minimum and
maximum of the surface radius.

\item[\code{sf\_centroid\_x}, \code{sf\_centroid\_y},
\code{sf\_centroid\_z}] The centre of the surface.  While the
quantities \code{sf\_origin\_*} denote the point from which the
radius of the surface is measured, the quantities
\code{sf\_centroid\_*} should contain the point which is
``logically'' the centre of the surface.  This quantity is supposed to
be a measure of the dipole moment of the surface.  One suggested
expression is $D^i = \int_S d\Omega\, x^i / A$.

\item[\code{sf\_quadrupole\_xx}, \code{sf\_quadrupole\_xy},
\code{sf\_quadrupole\_xz}, \code{sf\_quadrupole\_yy},
\code{sf\_quadrupole\_yz}, \code{sf\_quadrupole\_zz}] The
quadrupole moment of the surface. This should be the full quadrupole
moment and not a trace-free quantity.  One suggested expression is
$Q^{ij} = \int_S d\Omega\, y^i y^j / A$ with $y^i = x^i - D^i$.

\item[\code{sf\_min\_x}, \code{sf\_min\_y}, \code{sf\_min\_z},
\code{sf\_max\_x}, \code{sf\_max\_y}, \code{sf\_max\_z}] The
bounding box of the surface.

\end{description}
Note that the integral expressions are only suggestions which should
be adapted to whatever is natural for the stored surface.  The
suggested integral expressions also depend on the metric which is
used; this should be a ``natural'' metric for the surface.  E.g.\ for
apparent horizons, this might be the induced two-metric $q_{ij}$ from
the projection of the ADM three-metric $\gamma_{ij}$.



\subsection{Validity of Surface Data}

There is also an integer flag \code{valid} available.  Its
definition is up to the surface-providing thorn.  The following
interpretations are \emph{suggested}:
\begin{description}

\item[zero:] No surface is provided at this time step.

\item[negative:] No surface could be found at this time step.

\item[positive:] The surface data are valid.

\end{description}
Note that, if this flag is used, it is necessary to set this flag at
every iteration.  This flag is not automatically reset to zero.



\section{Surface Array Shape}

The number of grid points in the radius array \code{sf\_radius} is
determined by the parameters \code{ntheta} and \code{nphi}.  These
arrays exist for each surface.  (Internally, thorn SphericalSurface
stores all surfaces with the same array shape \code{maxntheta} and
\code{maxnphi}, so that the parameters  \code{ntheta} and
\code{nphi} must not be used for index calculations.  Use the
surfaces \code{lsh} instead.)  The surface array shape includes
ghost or boundary zones at the array edges.  These ghost zones have
the same size for all surfaces.

Note that because the radius arrays are stored with larger size
$\code{maxntheta} \times \code{maxnphi}$, the actual radius data
(of size
$\code{ntheta[surface\_number]} \times \code{nphi[surface\_number]}$
elements) is in general {\em not contiguous in memory.\/}
If you want to interpolate a SphericalSurface surface radius,
you need to either copy the radius data to a contiguous 2-D array of size
$\code{ntheta[surface\_number]} \times \code{nphi[surface\_number]}$,
or use an interpolator which supports such non-contiguous input arrays.
The AEIThorns/AEILocalInterp local interpolation thorn supports these
via the \code{input\_array\_strides} parameter-table option.
See the AEILocalInterp thorn guide for details.



\section{Surface Symmetries}

It is often the case that one uses symmetries to reduce the size of
the simulation domain, such as octant or quadrant mode.  Whenever a
symmetry plane intersects a surface, only part of the surface is
actually stored.  The user has to define the symmetries of each
surface in the parameter file via the parameters
\code{sf\_symmetry\_x}, \code{sf\_symmetry\_y}, and
\code{sf\_symmetry\_z}.  They indicate that a 
reflection symmetry exists in the corresponding direction.  The
surface origin is required to lie in the corresponding symmetry
planes.

Thorn SphericalSurface takes these symmetries into account when it
calculates the grid spacing and the origin of the surface coordinates
$\theta$ and $\phi$.



\section{Input and Output}

As the surfaces are stored as grid variables, the standard input and
output routines will work for them.  The standard visualisation tools
will be able to visualise them.  The surfaces will also automatically
be checkpointed and restored.



\subsection{Acknowledgements}

This thorn was suggested during meetings of the numerical relativity
group at the AEI.  Jonathan Thornburg provided many detailed and
useful suggestions.  Ed Evans, Carsten Gundlach, Ian Hawke, and Denis
Pollney contributed comments and suggestions.

% Do not delete next line
% END CACTUS THORNGUIDE



\section{Parameters} 


\parskip = 0pt

\setlength{\tableWidth}{160mm}

\setlength{\paraWidth}{\tableWidth}
\setlength{\descWidth}{\tableWidth}
\settowidth{\maxVarWidth}{fully contained}

\addtolength{\paraWidth}{-\maxVarWidth}
\addtolength{\paraWidth}{-\columnsep}
\addtolength{\paraWidth}{-\columnsep}
\addtolength{\paraWidth}{-\columnsep}

\addtolength{\descWidth}{-\columnsep}
\addtolength{\descWidth}{-\columnsep}
\addtolength{\descWidth}{-\columnsep}
\noindent \begin{tabular*}{\tableWidth}{|c|l@{\extracolsep{\fill}}r|}
\hline
\multicolumn{1}{|p{\maxVarWidth}}{origin\_x} & {\bf Scope:} private & REAL \\\hline
\multicolumn{3}{|p{\descWidth}|}{{\bf Description:}   {\em Origin for surface}} \\
\hline{\bf Range} & &  {\bf Default:} 0.0 \\\multicolumn{1}{|p{\maxVarWidth}|}{\centering *} & \multicolumn{2}{p{\paraWidth}|}{origin} \\\hline
\end{tabular*}

\vspace{0.5cm}\noindent \begin{tabular*}{\tableWidth}{|c|l@{\extracolsep{\fill}}r|}
\hline
\multicolumn{1}{|p{\maxVarWidth}}{origin\_y} & {\bf Scope:} private & REAL \\\hline
\multicolumn{3}{|p{\descWidth}|}{{\bf Description:}   {\em Origin for surface}} \\
\hline{\bf Range} & &  {\bf Default:} 0.0 \\\multicolumn{1}{|p{\maxVarWidth}|}{\centering *} & \multicolumn{2}{p{\paraWidth}|}{origin} \\\hline
\end{tabular*}

\vspace{0.5cm}\noindent \begin{tabular*}{\tableWidth}{|c|l@{\extracolsep{\fill}}r|}
\hline
\multicolumn{1}{|p{\maxVarWidth}}{origin\_z} & {\bf Scope:} private & REAL \\\hline
\multicolumn{3}{|p{\descWidth}|}{{\bf Description:}   {\em Origin for surface}} \\
\hline{\bf Range} & &  {\bf Default:} 0.0 \\\multicolumn{1}{|p{\maxVarWidth}|}{\centering *} & \multicolumn{2}{p{\paraWidth}|}{origin} \\\hline
\end{tabular*}

\vspace{0.5cm}\noindent \begin{tabular*}{\tableWidth}{|c|l@{\extracolsep{\fill}}r|}
\hline
\multicolumn{1}{|p{\maxVarWidth}}{radius} & {\bf Scope:} private & REAL \\\hline
\multicolumn{3}{|p{\descWidth}|}{{\bf Description:}   {\em Radius for surface}} \\
\hline{\bf Range} & &  {\bf Default:} 0.0 \\\multicolumn{1}{|p{\maxVarWidth}|}{\centering *} & \multicolumn{2}{p{\paraWidth}|}{radius} \\\hline
\end{tabular*}

\vspace{0.5cm}\noindent \begin{tabular*}{\tableWidth}{|c|l@{\extracolsep{\fill}}r|}
\hline
\multicolumn{1}{|p{\maxVarWidth}}{radius\_x} & {\bf Scope:} private & REAL \\\hline
\multicolumn{3}{|p{\descWidth}|}{{\bf Description:}   {\em Radius for surface}} \\
\hline{\bf Range} & &  {\bf Default:} 0.0 \\\multicolumn{1}{|p{\maxVarWidth}|}{\centering *} & \multicolumn{2}{p{\paraWidth}|}{radius} \\\hline
\end{tabular*}

\vspace{0.5cm}\noindent \begin{tabular*}{\tableWidth}{|c|l@{\extracolsep{\fill}}r|}
\hline
\multicolumn{1}{|p{\maxVarWidth}}{radius\_y} & {\bf Scope:} private & REAL \\\hline
\multicolumn{3}{|p{\descWidth}|}{{\bf Description:}   {\em Radius for surface}} \\
\hline{\bf Range} & &  {\bf Default:} 0.0 \\\multicolumn{1}{|p{\maxVarWidth}|}{\centering *} & \multicolumn{2}{p{\paraWidth}|}{radius} \\\hline
\end{tabular*}

\vspace{0.5cm}\noindent \begin{tabular*}{\tableWidth}{|c|l@{\extracolsep{\fill}}r|}
\hline
\multicolumn{1}{|p{\maxVarWidth}}{radius\_z} & {\bf Scope:} private & REAL \\\hline
\multicolumn{3}{|p{\descWidth}|}{{\bf Description:}   {\em Radius for surface}} \\
\hline{\bf Range} & &  {\bf Default:} 0.0 \\\multicolumn{1}{|p{\maxVarWidth}|}{\centering *} & \multicolumn{2}{p{\paraWidth}|}{radius} \\\hline
\end{tabular*}

\vspace{0.5cm}\noindent \begin{tabular*}{\tableWidth}{|c|l@{\extracolsep{\fill}}r|}
\hline
\multicolumn{1}{|p{\maxVarWidth}}{set\_elliptic} & {\bf Scope:} private & BOOLEAN \\\hline
\multicolumn{3}{|p{\descWidth}|}{{\bf Description:}   {\em Place surface at a certain radius}} \\
\hline & & {\bf Default:} no \\\hline
\end{tabular*}

\vspace{0.5cm}\noindent \begin{tabular*}{\tableWidth}{|c|l@{\extracolsep{\fill}}r|}
\hline
\multicolumn{1}{|p{\maxVarWidth}}{set\_spherical} & {\bf Scope:} private & BOOLEAN \\\hline
\multicolumn{3}{|p{\descWidth}|}{{\bf Description:}   {\em Place surface at a certain radius}} \\
\hline & & {\bf Default:} no \\\hline
\end{tabular*}

\vspace{0.5cm}\noindent \begin{tabular*}{\tableWidth}{|c|l@{\extracolsep{\fill}}r|}
\hline
\multicolumn{1}{|p{\maxVarWidth}}{auto\_res} & {\bf Scope:} restricted & BOOLEAN \\\hline
\multicolumn{3}{|p{\descWidth}|}{{\bf Description:}   {\em Automatically determine resolution according to radius and Cartesian resolution}} \\
\hline & & {\bf Default:} no \\\hline
\end{tabular*}

\vspace{0.5cm}\noindent \begin{tabular*}{\tableWidth}{|c|l@{\extracolsep{\fill}}r|}
\hline
\multicolumn{1}{|p{\maxVarWidth}}{auto\_res\_grid} & {\bf Scope:} restricted & KEYWORD \\\hline
\multicolumn{3}{|p{\descWidth}|}{{\bf Description:}   {\em Choose resolution according to how grids overlap}} \\
\hline{\bf Range} & &  {\bf Default:} fully contained \\\multicolumn{1}{|p{\maxVarWidth}|}{\centering fully contained} & \multicolumn{2}{p{\paraWidth}|}{SF must be fully contained in Cartesian grid} \\\multicolumn{1}{|p{\maxVarWidth}|}{\centering overlap} & \multicolumn{2}{p{\paraWidth}|}{SF overlaps with grid} \\\multicolumn{1}{|p{\maxVarWidth}|}{\centering multipatch} & \multicolumn{2}{p{\paraWidth}|}{SF potentially overlaps with a spherical mutipatch grid} \\\hline
\end{tabular*}

\vspace{0.5cm}\noindent \begin{tabular*}{\tableWidth}{|c|l@{\extracolsep{\fill}}r|}
\hline
\multicolumn{1}{|p{\maxVarWidth}}{auto\_res\_ratio} & {\bf Scope:} restricted & REAL \\\hline
\multicolumn{3}{|p{\descWidth}|}{{\bf Description:}   {\em Multiplicative factor by which we want to scale the resolution with respect to Cartesian resolution}} \\
\hline{\bf Range} & &  {\bf Default:} 2.0 \\\multicolumn{1}{|p{\maxVarWidth}|}{\centering 0:*} & \multicolumn{2}{p{\paraWidth}|}{} \\\hline
\end{tabular*}

\vspace{0.5cm}\noindent \begin{tabular*}{\tableWidth}{|c|l@{\extracolsep{\fill}}r|}
\hline
\multicolumn{1}{|p{\maxVarWidth}}{maxnphi} & {\bf Scope:} restricted & INT \\\hline
\multicolumn{3}{|p{\descWidth}|}{{\bf Description:}   {\em Maximum number of grid points in the phi direction}} \\
\hline{\bf Range} & &  {\bf Default:} 38 \\\multicolumn{1}{|p{\maxVarWidth}|}{\centering 0:*} & \multicolumn{2}{p{\paraWidth}|}{} \\\hline
\end{tabular*}

\vspace{0.5cm}\noindent \begin{tabular*}{\tableWidth}{|c|l@{\extracolsep{\fill}}r|}
\hline
\multicolumn{1}{|p{\maxVarWidth}}{maxntheta} & {\bf Scope:} restricted & INT \\\hline
\multicolumn{3}{|p{\descWidth}|}{{\bf Description:}   {\em Maximum number of grid points in the theta direction}} \\
\hline{\bf Range} & &  {\bf Default:} 19 \\\multicolumn{1}{|p{\maxVarWidth}|}{\centering 0:*} & \multicolumn{2}{p{\paraWidth}|}{} \\\hline
\end{tabular*}

\vspace{0.5cm}\noindent \begin{tabular*}{\tableWidth}{|c|l@{\extracolsep{\fill}}r|}
\hline
\multicolumn{1}{|p{\maxVarWidth}}{name} & {\bf Scope:} restricted & STRING \\\hline
\multicolumn{3}{|p{\descWidth}|}{{\bf Description:}   {\em User friendly name of spherical surface}} \\
\hline{\bf Range} & &  {\bf Default:} (none) \\\multicolumn{1}{|p{\maxVarWidth}|}{\centering } & \multicolumn{2}{p{\paraWidth}|}{none set} \\\multicolumn{1}{|p{\maxVarWidth}|}{\centering .*} & \multicolumn{2}{p{\paraWidth}|}{any string} \\\hline
\end{tabular*}

\vspace{0.5cm}\noindent \begin{tabular*}{\tableWidth}{|c|l@{\extracolsep{\fill}}r|}
\hline
\multicolumn{1}{|p{\maxVarWidth}}{nghostsphi} & {\bf Scope:} restricted & INT \\\hline
\multicolumn{3}{|p{\descWidth}|}{{\bf Description:}   {\em Number of ghost zones in the phi direction}} \\
\hline{\bf Range} & &  {\bf Default:} 2 \\\multicolumn{1}{|p{\maxVarWidth}|}{\centering 0:*} & \multicolumn{2}{p{\paraWidth}|}{} \\\hline
\end{tabular*}

\vspace{0.5cm}\noindent \begin{tabular*}{\tableWidth}{|c|l@{\extracolsep{\fill}}r|}
\hline
\multicolumn{1}{|p{\maxVarWidth}}{nghoststheta} & {\bf Scope:} restricted & INT \\\hline
\multicolumn{3}{|p{\descWidth}|}{{\bf Description:}   {\em Number of ghost zones in the theta direction}} \\
\hline{\bf Range} & &  {\bf Default:} 2 \\\multicolumn{1}{|p{\maxVarWidth}|}{\centering 0:*} & \multicolumn{2}{p{\paraWidth}|}{} \\\hline
\end{tabular*}

\vspace{0.5cm}\noindent \begin{tabular*}{\tableWidth}{|c|l@{\extracolsep{\fill}}r|}
\hline
\multicolumn{1}{|p{\maxVarWidth}}{nphi} & {\bf Scope:} restricted & INT \\\hline
\multicolumn{3}{|p{\descWidth}|}{{\bf Description:}   {\em Number of grid points in the phi direction}} \\
\hline{\bf Range} & &  {\bf Default:} 38 \\\multicolumn{1}{|p{\maxVarWidth}|}{\centering 0:*} & \multicolumn{2}{p{\paraWidth}|}{must be at least 3*nghostsphi} \\\hline
\end{tabular*}

\vspace{0.5cm}\noindent \begin{tabular*}{\tableWidth}{|c|l@{\extracolsep{\fill}}r|}
\hline
\multicolumn{1}{|p{\maxVarWidth}}{nsurfaces} & {\bf Scope:} restricted & INT \\\hline
\multicolumn{3}{|p{\descWidth}|}{{\bf Description:}   {\em Number of surfaces}} \\
\hline{\bf Range} & &  {\bf Default:} (none) \\\multicolumn{1}{|p{\maxVarWidth}|}{\centering 0:42} & \multicolumn{2}{p{\paraWidth}|}{} \\\hline
\end{tabular*}

\vspace{0.5cm}\noindent \begin{tabular*}{\tableWidth}{|c|l@{\extracolsep{\fill}}r|}
\hline
\multicolumn{1}{|p{\maxVarWidth}}{ntheta} & {\bf Scope:} restricted & INT \\\hline
\multicolumn{3}{|p{\descWidth}|}{{\bf Description:}   {\em Number of grid points in the theta direction}} \\
\hline{\bf Range} & &  {\bf Default:} 19 \\\multicolumn{1}{|p{\maxVarWidth}|}{\centering 0:*} & \multicolumn{2}{p{\paraWidth}|}{must be at least 3*nghoststheta} \\\hline
\end{tabular*}

\vspace{0.5cm}\noindent \begin{tabular*}{\tableWidth}{|c|l@{\extracolsep{\fill}}r|}
\hline
\multicolumn{1}{|p{\maxVarWidth}}{symmetric\_x} & {\bf Scope:} restricted & BOOLEAN \\\hline
\multicolumn{3}{|p{\descWidth}|}{{\bf Description:}   {\em Reflection symmetry in the x direction}} \\
\hline & & {\bf Default:} no \\\hline
\end{tabular*}

\vspace{0.5cm}\noindent \begin{tabular*}{\tableWidth}{|c|l@{\extracolsep{\fill}}r|}
\hline
\multicolumn{1}{|p{\maxVarWidth}}{symmetric\_y} & {\bf Scope:} restricted & BOOLEAN \\\hline
\multicolumn{3}{|p{\descWidth}|}{{\bf Description:}   {\em Reflection symmetry in the y direction}} \\
\hline & & {\bf Default:} no \\\hline
\end{tabular*}

\vspace{0.5cm}\noindent \begin{tabular*}{\tableWidth}{|c|l@{\extracolsep{\fill}}r|}
\hline
\multicolumn{1}{|p{\maxVarWidth}}{symmetric\_z} & {\bf Scope:} restricted & BOOLEAN \\\hline
\multicolumn{3}{|p{\descWidth}|}{{\bf Description:}   {\em Reflection symmetry in the z direction}} \\
\hline & & {\bf Default:} no \\\hline
\end{tabular*}

\vspace{0.5cm}\noindent \begin{tabular*}{\tableWidth}{|c|l@{\extracolsep{\fill}}r|}
\hline
\multicolumn{1}{|p{\maxVarWidth}}{verbose} & {\bf Scope:} restricted & BOOLEAN \\\hline
\multicolumn{3}{|p{\descWidth}|}{{\bf Description:}   {\em Shall I be verbose?}} \\
\hline & & {\bf Default:} no \\\hline
\end{tabular*}

\vspace{0.5cm}\parskip = 10pt 

\section{Interfaces} 


\parskip = 0pt

\vspace{3mm} \subsection*{General}

\noindent {\bf Implements}: 

sphericalsurface
\vspace{2mm}

\noindent {\bf Inherits}: 

grid
\vspace{2mm}
\subsection*{Grid Variables}
\vspace{5mm}\subsubsection{PRIVATE GROUPS}

\vspace{5mm}

\begin{tabular*}{150mm}{|c|c@{\extracolsep{\fill}}|rl|} \hline 
~ {\bf Group Names} ~ & ~ {\bf Variable Names} ~  &{\bf Details} ~ & ~\\ 
\hline 
sf\_coordinate\_estimators &  & compact & 0 \\ 
 & sf\_delta\_theta\_estimate & description & Surface coordinate estimators \\ 
 & sf\_delta\_phi\_estimate & dimensions & 0 \\ 
 &  & distribution & CONSTANT \\ 
 &  & group type & SCALAR \\ 
 &  & timelevels & 1 \\ 
 &  & vararray\_size & nsurfaces \\ 
 &  & variable type & REAL \\ 
\hline 
\end{tabular*} 


\vspace{5mm}\subsubsection{PUBLIC GROUPS}

\vspace{5mm}

\begin{tabular*}{150mm}{|c|c@{\extracolsep{\fill}}|rl|} \hline 
~ {\bf Group Names} ~ & ~ {\bf Variable Names} ~  &{\bf Details} ~ & ~\\ 
\hline 
sf\_active & sf\_active & compact & 0 \\ 
 &  & dimensions & 0 \\ 
 &  & distribution & CONSTANT \\ 
 &  & group type & SCALAR \\ 
 &  & timelevels & 1 \\ 
 &  & vararray\_size & nsurfaces \\ 
 &  & variable type & INT \\ 
\hline 
sf\_valid & sf\_valid & compact & 0 \\ 
 &  & dimensions & 0 \\ 
 &  & distribution & CONSTANT \\ 
 &  & group type & SCALAR \\ 
 &  & timelevels & 1 \\ 
 &  & vararray\_size & nsurfaces \\ 
 &  & variable type & INT \\ 
\hline 
sf\_info &  & compact & 0 \\ 
 & sf\_area & description & Surface information \\ 
 & sf\_mean\_radius & dimensions & 0 \\ 
 & sf\_centroid\_x & distribution & CONSTANT \\ 
 & sf\_centroid\_y & group type & SCALAR \\ 
 & sf\_centroid\_z & timelevels & 1 \\ 
 & sf\_quadrupole\_xx & vararray\_size & nsurfaces \\ 
 & sf\_quadrupole\_xy & variable type & REAL \\ 
\hline 
sf\_minreflevel & sf\_minreflevel & compact & 0 \\ 
 &  & dimensions & 0 \\ 
 &  & distribution & CONSTANT \\ 
 &  & group type & SCALAR \\ 
 &  & timelevels & 1 \\ 
 &  & vararray\_size & nsurfaces \\ 
 &  & variable type & INT \\ 
\hline 
sf\_maxreflevel & sf\_maxreflevel & compact & 0 \\ 
 &  & dimensions & 0 \\ 
 &  & distribution & CONSTANT \\ 
 &  & group type & SCALAR \\ 
 &  & timelevels & 1 \\ 
 &  & vararray\_size & nsurfaces \\ 
 &  & variable type & INT \\ 
\hline 
sf\_radius & sf\_radius & compact & 0 \\ 
 &  & dimensions & 2 \\ 
 &  & distribution & CONSTANT \\ 
 &  & group type & ARRAY \\ 
 &  & size & MAXNTHETA \\ 
& ~ & size & MAXNPHI \\ 
 &  & timelevels & 1 \\ 
 &  & vararray\_size & nsurfaces \\ 
 &  & variable type & REAL \\ 
\hline 
\end{tabular*} 



\vspace{5mm}
\vspace{5mm}

\begin{tabular*}{150mm}{|c|c@{\extracolsep{\fill}}|rl|} \hline 
~ {\bf Group Names} ~ & ~ {\bf Variable Names} ~  &{\bf Details} ~ & ~ \\ 
\hline 
sf\_origin &  & compact & 0 \\ 
 & sf\_origin\_x & dimensions & 0 \\ 
 & sf\_origin\_y & distribution & CONSTANT \\ 
 & sf\_origin\_z & group type & SCALAR \\ 
 &  & timelevels & 1 \\ 
 &  & vararray\_size & nsurfaces \\ 
 &  & variable type & REAL \\ 
\hline 
sf\_coordinate\_descriptors &  & compact & 0 \\ 
 & sf\_origin\_theta & description & Surface coordinate descriptors \\ 
 & sf\_origin\_phi & dimensions & 0 \\ 
 & sf\_delta\_theta & distribution & CONSTANT \\ 
 & sf\_delta\_phi & group type & SCALAR \\ 
 &  & timelevels & 1 \\ 
 &  & vararray\_size & nsurfaces \\ 
 &  & variable type & REAL \\ 
\hline 
sf\_shape\_descriptors &  & compact & 0 \\ 
 & sf\_ntheta & description & Surface shape descriptors \\ 
 & sf\_nphi & dimensions & 0 \\ 
 & sf\_nghoststheta & distribution & CONSTANT \\ 
 & sf\_nghostsphi & group type & SCALAR \\ 
 &  & timelevels & 1 \\ 
 &  & vararray\_size & nsurfaces \\ 
 &  & variable type & INT \\ 
\hline 
\end{tabular*} 



\vspace{5mm}

\noindent {\bf Provides}: 



sf\_IdFromName to 
\vspace{2mm}\parskip = 10pt 

\section{Schedule} 


\parskip = 0pt


\noindent This section lists all the variables which are assigned storage by thorn CactusNumerical/SphericalSurface.  Storage can either last for the duration of the run ({\bf Always} means that if this thorn is activated storage will be assigned, {\bf Conditional} means that if this thorn is activated storage will be assigned for the duration of the run if some condition is met), or can be turned on for the duration of a schedule function.


\subsection*{Storage}

\hspace{5mm}

 \begin{tabular*}{160mm}{ll} 

{\bf Always:}&  ~ \\ 
 sf\_active & ~\\ 
 sf\_valid & ~\\ 
 sf\_info & ~\\ 
 sf\_radius sf\_origin & ~\\ 
 sf\_coordinate\_descriptors & ~\\ 
 sf\_coordinate\_estimators & ~\\ 
 sf\_shape\_descriptors & ~\\ 
 sf\_minreflevel & ~\\ 
 sf\_maxreflevel & ~\\ 
~ & ~\\ 
\end{tabular*} 


\subsection*{Scheduled Functions}
\vspace{5mm}

\noindent {\bf CCTK\_BASEGRID} 

\hspace{5mm} sphericalsurface\_setupres 

\hspace{5mm}{\it set surface resolution automatically } 


\hspace{5mm}

 \begin{tabular*}{160mm}{cll} 
~ & After:  & spatialcoordinates \\ 
~& ~ &correctcoordinates\\ 
~ & Before:  & sphericalsurface\_setup \\ 
~ & Language:  & c \\ 
~ & Options:  & global \\ 
~& ~ &loop-local\\ 
~ & Type:  & function \\ 
~ & Writes:  & sphericalsurface::sf\_coordinate\_estimators(everywhere) \\ 
~& ~ &sphericalsurface::sf\_minreflevel(everywhere)\\ 
~& ~ &sphericalsurface::sf\_maxreflevel(everywhere)\\ 
\end{tabular*} 


\vspace{5mm}

\noindent {\bf CCTK\_BASEGRID} 

\hspace{5mm} sphericalsurface\_setup 

\hspace{5mm}{\it calculate surface coordinate descriptors } 


\hspace{5mm}

 \begin{tabular*}{160mm}{cll} 
~ & Language:  & c \\ 
~ & Options:  & global \\ 
~ & Reads:  & sphericalsurface::sf\_coordinate\_estimators \\ 
~& ~ &sphericalsurface::sf\_minreflevel\\ 
~& ~ &sphericalsurface::sf\_maxreflevel\\ 
~ & Type:  & function \\ 
~ & Writes:  & sphericalsurface::sf\_shape\_descriptors(everywhere) \\ 
~& ~ &sphericalsurface::sf\_coordinate\_descriptors(everywhere)\\ 
~& ~ &sphericalsurface::sf\_active(everywhere)\\ 
~& ~ &sphericalsurface::sf\_valid(everywhere)\\ 
\end{tabular*} 


\vspace{5mm}

\noindent {\bf CCTK\_BASEGRID} 

\hspace{5mm} sphericalsurface\_set 

\hspace{5mm}{\it set surface radii to be used for initial setup in other thorns } 


\hspace{5mm}

 \begin{tabular*}{160mm}{cll} 
~ & After:  & sphericalsurface\_setup \\ 
~ & Before:  & sphericalsurface\_hasbeenset \\ 
~ & Language:  & c \\ 
~ & Options:  & global \\ 
~ & Reads:  & sphericalsurface::sf\_shape\_descriptors \\ 
~& ~ &sphericalsurface::sf\_coordinate\_descriptors\\ 
~ & Type:  & function \\ 
~ & Writes:  & sphericalsurface::sf\_active(everywhere) \\ 
~& ~ &sphericalsurface::sf\_valid(everywhere)\\ 
~& ~ &sphericalsurface::sf\_info(everywhere)\\ 
~& ~ &sphericalsurface::sf\_origin(everywhere)\\ 
~& ~ &sphericalsurface::sf\_radius(everywhere)\\ 
\end{tabular*} 


\vspace{5mm}

\noindent {\bf CCTK\_BASEGRID} 

\hspace{5mm} sphericalsurface\_hasbeenset 

\hspace{5mm}{\it set the spherical surfaces before this group, and use it afterwards } 


\hspace{5mm}

 \begin{tabular*}{160mm}{cll} 
~ & Type:  & group \\ 
\end{tabular*} 


\vspace{5mm}

\noindent {\bf CCTK\_POSTSTEP} 

\hspace{5mm} sphericalsurface\_set 

\hspace{5mm}{\it set surface radii } 


\hspace{5mm}

 \begin{tabular*}{160mm}{cll} 
~ & Before:  & sphericalsurface\_hasbeenset \\ 
~ & Language:  & c \\ 
~ & Options:  & global \\ 
~ & Type:  & function \\ 
\end{tabular*} 


\vspace{5mm}

\noindent {\bf CCTK\_POSTSTEP} 

\hspace{5mm} sphericalsurface\_hasbeenset 

\hspace{5mm}{\it set the spherical surfaces before this group, and use it afterwards } 


\hspace{5mm}

 \begin{tabular*}{160mm}{cll} 
~ & Type:  & group \\ 
\end{tabular*} 


\vspace{5mm}

\noindent {\bf SphericalSurface\_HasBeenSet} 

\hspace{5mm} sphericalsurface\_checkstate 

\hspace{5mm}{\it test the state of the spherical surfaces } 


\hspace{5mm}

 \begin{tabular*}{160mm}{cll} 
~ & Language:  & c \\ 
~ & Options:  & global \\ 
~ & Reads:  & sphericalsurface::sf\_valid \\ 
~& ~ &sphericalsurface::sf\_active\\ 
~ & Type:  & function \\ 
\end{tabular*} 


\vspace{5mm}

\noindent {\bf CCTK\_PARAMCHECK} 

\hspace{5mm} sphericalsurface\_paramcheck 

\hspace{5mm}{\it check that all surface names are unique } 


\hspace{5mm}

 \begin{tabular*}{160mm}{cll} 
~ & Language:  & c \\ 
~ & Options:  & global \\ 
~ & Type:  & function \\ 
\end{tabular*} 



\vspace{5mm}\parskip = 10pt 
\end{document}
