\documentclass{article}

% Use the Cactus ThornGuide style file
% (Automatically used from Cactus distribution, if you have a 
%  thorn without the Cactus Flesh download this from the Cactus
%  homepage at www.cactuscode.org)
\usepackage{../../../../../doc/latex/cactus}

\newlength{\tableWidth} \newlength{\maxVarWidth} \newlength{\paraWidth} \newlength{\descWidth} \begin{document}

\title{EllSOR}
\author{Joan Masso}
\date{$ $Date$ $}

\maketitle

% Do not delete next line
% START CACTUS THORNGUIDE

\begin{abstract}
{\tt EllSOR} provides 3D elliptic solvers for the various
classes of elliptic problems defined in {\tt EllBase}. {\tt EllSOR} is 
based on the successive over relaxation algorithm. It is called by the
interfaces provided in {\tt EllBase}.
\end{abstract}

\section{Purpose}
The purpose of this thorn is to provide a simple and straightforward
3D elliptic solver: not to be used by production but to demonstrate key
features of the elliptic infrastructure.

This thorn provides
 \begin{enumerate}
  \item No Pizza
  \item No Wine
  \item peace
 \end{enumerate}

\section{Technical Details}
This thorn supports three elliptic problem classes: {\bf LinFlat} for 
a standard 3D cartesian Laplace operator, using the standard 7-point
computational molecule. {\bf LinMetric} for a Laplace operator derived
from the metric, using 19-point stencil. {\bf LinConfMetric} for a
Laplace operator derived from the metric and a conformal factor, using 
a 19-point stencil. The code of the solvers differs for the classes
and is explained in the following section. 

In general, a stencil variable needs to be set for each of the
direction relative to the central gridpoint. These variables are
called {\tt ac}, {\tt ae}, {\tt aw}, {\tt an}, {\tt as}, {\tt at}, {\tt ab}, {\tt
ane}, {\tt anw}, {\tt ase}, {\tt asw}, {\tt ate}, {\tt atw}, {\tt abe}, {\tt
abw}, {\tt atn}, {\tt ats}, {\tt abn}, {\tt asb}, where ``{\tt ac}'' =
a-central, ``{\tt t}'' = top, ``{\tt b}'' = bottom, ``{\tt n,s,w,e}'' = north, south, west, east

\subsection{{\bf LinFlat}}
For this class we employ the the 7-point stencil based on {\tt at,ab,
aw, ae, an, as} only. These values are constant at each gridpoint.

\subsection{{\bf LinMetric}}
For this class the standard 19-point stencil is initialized, taken the 
underlying metric into account. The values for the stencil function
differ at each gridpoints.

\subsection{{\bf LinConfMetric}}
For this class the standard 19-point stencil is initialized, taken the 
underlying metric and its conformal factor into account. The values
for the stencil function differ at each gridpoints.

\section{Comments}
The sizes of the arrays {\tt Mlinear} for the coefficient matrix and
{\tt Nsource} are passed in the solver. A storage flag is set if these 
variables are of a sized greater 1. In this case, the array can be
accessed.

%\section{My own section}

% Do not delete next line
% END CACTUS THORNGUIDE



\section{Parameters} 


\parskip = 0pt

\setlength{\tableWidth}{160mm}

\setlength{\paraWidth}{\tableWidth}
\setlength{\descWidth}{\tableWidth}
\settowidth{\maxVarWidth}{elliptic\_verbose}

\addtolength{\paraWidth}{-\maxVarWidth}
\addtolength{\paraWidth}{-\columnsep}
\addtolength{\paraWidth}{-\columnsep}
\addtolength{\paraWidth}{-\columnsep}

\addtolength{\descWidth}{-\columnsep}
\addtolength{\descWidth}{-\columnsep}
\addtolength{\descWidth}{-\columnsep}
\noindent \begin{tabular*}{\tableWidth}{|c|l@{\extracolsep{\fill}}r|}
\hline
\multicolumn{1}{|p{\maxVarWidth}}{elliptic\_verbose} & {\bf Scope:} shared from ELLBASE & KEYWORD \\\hline
\end{tabular*}

\vspace{0.5cm}\parskip = 10pt 

\section{Interfaces} 


\parskip = 0pt

\vspace{3mm} \subsection*{General}

\noindent {\bf Implements}: 

ellsor
\vspace{2mm}

\noindent {\bf Inherits}: 

ellbase

boundary
\vspace{2mm}

\vspace{5mm}

\noindent {\bf Uses header}: 

EllBase.h

Ell\_DBstructure.h

Boundary.h

Symmetry.h
\vspace{2mm}\parskip = 10pt 

\section{Schedule} 


\parskip = 0pt


\noindent This section lists all the variables which are assigned storage by thorn CactusElliptic/EllSOR.  Storage can either last for the duration of the run ({\bf Always} means that if this thorn is activated storage will be assigned, {\bf Conditional} means that if this thorn is activated storage will be assigned for the duration of the run if some condition is met), or can be turned on for the duration of a schedule function.


\subsection*{Storage}NONE
\subsection*{Scheduled Functions}
\vspace{5mm}

\noindent {\bf CCTK\_BASEGRID} 

\hspace{5mm} ellsor\_register 

\hspace{5mm}{\it register the sor solvers } 


\hspace{5mm}

 \begin{tabular*}{160mm}{cll} 
~ & Language:  & c \\ 
~ & Type:  & function \\ 
\end{tabular*} 



\vspace{5mm}\parskip = 10pt 
\end{document}
