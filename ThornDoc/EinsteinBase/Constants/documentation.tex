\documentclass{article}

% Use the Cactus ThornGuide style file
% (Automatically used from Cactus distribution, if you have a 
%  thorn without the Cactus Flesh download this from the Cactus
%  homepage at www.cactuscode.org)
\usepackage{../../../../../doc/latex/cactus}

\newlength{\tableWidth} \newlength{\maxVarWidth} \newlength{\paraWidth} \newlength{\descWidth} \begin{document}

\title{Constants}
\author{Frank L\"offler (knarf@cct.lsu.edu)}
\date{$ $Date$ $}

\maketitle

% Do not delete next line
% START CACTUS THORNGUIDE

\begin{abstract}
Thorn to provide macros for commonly used constants like the speed of light,
the gravitational constant or the solar mass.
\end{abstract}

\section{Comments}
 Because the number of defines could grow, I define some naming conventions:
 \begin{itemize}
  \item Names have to start with \textbf{CONSTANT\_}.
  \item Afterwards a typical name is following.
  \item A \_ is coming next.
  \item Either SI, cgi or C (for Cactus) is the last part. This specifies the
        system of units used here (SI, cgi or Cactus units).
 \end{itemize}
\section{Provided Constants}
 \begin{itemize}
  \item $c$ : the speed of light
  \item $G$ : the gravitational constant
  \item $M_{\odot}$ : the solar mass
 \end{itemize}
 
% Do not delete next line
% END CACTUS THORNGUIDE


\section{Parameters} 


\parskip = 0pt
\parskip = 10pt 

\section{Interfaces} 


\parskip = 0pt

\vspace{3mm} \subsection*{General}

\noindent {\bf Implements}: 

constants
\vspace{2mm}

\vspace{5mm}

\noindent {\bf Adds header}: 



constants.h
\vspace{2mm}\parskip = 10pt 

\section{Schedule} 


\parskip = 0pt


\noindent This section lists all the variables which are assigned storage by thorn EinsteinBase/Constants.  Storage can either last for the duration of the run ({\bf Always} means that if this thorn is activated storage will be assigned, {\bf Conditional} means that if this thorn is activated storage will be assigned for the duration of the run if some condition is met), or can be turned on for the duration of a schedule function.


\subsection*{Storage}NONE

\vspace{5mm}\parskip = 10pt 
\end{document}
